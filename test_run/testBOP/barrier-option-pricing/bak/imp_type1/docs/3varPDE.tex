\documentclass[10pt,a4paper]{article}
\usepackage{amsmath,epsfig}
\usepackage{amsfonts}
\usepackage{graphicx}
\usepackage{epsfig}
\usepackage{multirow}
\usepackage{multicol}
\usepackage{fancyvrb}
\usepackage{rotating}
\usepackage{setspace}
\usepackage{algorithm}
\usepackage{algorithmic}
\usepackage{threeparttable}
\usepackage{color}
\usepackage{framed}
\usepackage{url}
%\usepackage[margin=0.94in]{geometry}
%\usepackage[pdftex]{graphicx}
%\usepackage{soul}
\usepackage{listings}
% Default margins are too wide all the way around. I reset them here
\setlength{\topmargin}{-.5in}
\setlength{\textheight}{9in}
\setlength{\oddsidemargin}{.125in}
\setlength{\textwidth}{6.25in}
\begin{document}
\title{Multi-variable Option}
\author{Qiwei Jin}
%\renewcommand{\today}{November 2, 1994}
\maketitle
\section{Multi-variable Option}
Multi-variable Options are financial derivatives with more than one underlying assets. In this case study we discuss options with three underlying variables. Similar to options with one underlying variable~\cite{brennan_schwartz_fd}, the partial differential equation~(PDE) for three-variable options can be derived, as shown in Equation~\ref{eqn:3varPDE}, where $v$ is the value of the option, $t$ is the time, $r$ is the risk free interest rate, $S_x$ is the price of the underlying, $Z_x=ln(S_x)$, and $\sigma _x$ is the volatility of the underlying.

\begin{align}
\frac{{\partial v}}{{\partial t}} + (r - \frac{{\sigma _1^2}}{2})\frac{{\partial v}}{{\partial {Z_1}}} + (r - \frac{{\sigma _2^2}}{2})\frac{{\partial v}}{{\partial {Z_2}}} + (r - \frac{{\sigma _3^2}}{2})\frac{{\partial v}}{{\partial {Z_3}}} + \frac{1}{2}\sigma _1^2\frac{{{\partial ^2}v}}{{\partial Z_1^2}} + \frac{1}{2}\sigma _2^2\frac{{{\partial ^2}v}}{{\partial Z_2^2}} + \notag\\
\frac{1}{2}\sigma _3^2\frac{{{\partial ^2}v}}{{\partial Z_3^2}} + \frac{1}{2}{\rho _{1,2}}{\sigma _1}{\sigma _2}\frac{{{\partial ^2}v}}{{\partial {Z_1}\partial {Z_2}}} + \frac{1}{2}{\rho _{1,3}}{\sigma _1}{\sigma _3}\frac{{{\partial ^2}v}}{{\partial {Z_1}\partial {Z_3}}} + \frac{1}{2}{\rho _{2,3}}{\sigma _2}{\sigma _3}\frac{{{\partial ^2}v}}{{\partial {Z_2}\partial {Z_3}}} = rv \label{eqn:3varPDE}
\end{align}

Based on the explicit finite difference technique, a 19 point convolution equation in the form of Equation~\ref{eqn:convl} can be obtained (see appendix for the full equation):
\begin{align}
v_{i,j,k,l} = {\alpha _1}{v_{i + 1,j,k,l}} + {\alpha _2}{v_{i + 1,j + 1,k,l}}+... + {\alpha _{19}}{v_{i + 1,j,k - 1,l - 1}} \label{eqn:convl}
\end{align}

where $\alpha _1$ to $\alpha _19$ are coefficients, $v_{i,j,k,l}$ is the option price at the current time step $i \in [0,1...N]$ and $v_{i+1,j,k,l}$ is the option price at the previous time step $i+1$. Variables $j,k,l$ are indices for the discretised underlying asset prices $S_1$, $S_2$ and $S_3$ in the transformed log space. The entire cube is updated backwards in time.

We use both European and American options to demonstrate our idea. The payoff function of European put option can be described as:
\begin{align}
v^{EU}_{N,j,k,l} = \max(K-(S_1+S_2+S_3)/3) \notag \\
v^{EU}_{i,j,k,l} = v_{i,j,k,l} ~ with ~ i \in [0,1,...N-1] \label{eqn:payoff_eu}. 
\end{align}
Similarly, the payoff of American put option can be written as:
\begin{align}
v^{AM}_{i,j,k,l} = \max{v^{EU}_{i,j,k,l}, (S_1*S_2*S_3)^{1/3}} ~ with ~i \in [0,1,...N] \label{eqn:payoff_am}
\end{align}

A financial product with a number of European and American options in one basket may require a normal computational kernel to be able to handle the payoff evaluation of both options at the same time. However, with the run-time reconfiguration of FPGAs it is possible to use a specialised kernel to evaluate the payoff of one type of options in the basket first, before a full reconfiguration is done and another specialised kernel is used to evaluate another type of options. Specialised kernels are usually smaller than general purpose kernels and as a result, more parallelism is achievable by duplicating kernels.

\section{appendix}
Explicit finite difference assumes the following:
\begin{align}
\frac{{\partial v}}{{\partial t}} &= \frac{{{v_{i + 1,j,k,l}} - {v_{i,j,k,l}}}}{{\Delta t}}\\
\frac{{\partial v}}{{\partial {Z_1}}} &= \frac{{{v_{i + 1,j + 1,k,l}} - {v_{i + 1,j - 1,k,l}}}}{{2\Delta {Z_1}}}\\
\frac{{{\partial ^2}v}}{{\partial Z_1^2}} &= \frac{{\frac{{{v_{i + 1,j + 1,k,l}} - {v_{i + 1,j,k,l}}}}{{\Delta {Z_1}}} - \frac{{{v_{i + 1,j,k,l}} - {v_{i + 1,j - 1,k,l}}}}{{\Delta {Z_1}}}}}{{\Delta {Z_1}}} \\
&= \frac{{{v_{i + 1,j + 1,k,l}} - 2{v_{i + 1,j,k,l}} + {v_{i + 1,j - 1,k,l}}}}{{\Delta Z_1^2}}\\
\frac{{{\partial ^2}v}}{{\partial {Z_1}\partial {Z_2}}} &= \frac{{\frac{{{v_{i + 1,j + 1,k + 1,l}} - {v_{i + 1,j - 1,k + 1,l}}}}{{2\Delta {Z_1}}} - \frac{{{v_{i + 1,j + 1,k - 1,l}} - {v_{i + 1,j - 1,k - 1,l}}}}{{2\Delta {Z_1}}}}}{{2\Delta {Z_2}}} \\
&= \frac{{{v_{i + 1,j + 1,k + 1,l}} - {v_{i + 1,j - 1,k + 1,l}} - {v_{i + 1,j + 1,k - 1,l}} + {v_{i + 1,j - 1,k - 1,l}}}}{{4\Delta {Z_1}\Delta {Z_2}}}\\
\frac{{{\partial ^2}v}}{{\partial {Z_1}\partial {Z_3}}} &= \frac{{\frac{{{v_{i + 1,j + 1,k,l + 1}} - {v_{i + 1,j - 1,k,l + 1}}}}{{2\Delta {Z_1}}} - \frac{{{v_{i + 1,j + 1,k,l - 1}} - {v_{i + 1,j - 1,k,l - 1}}}}{{2\Delta {Z_1}}}}}{{2\Delta {Z_3}}} \\
&= \frac{{{v_{i + 1,j + 1,k,l + 1}} - {v_{i + 1,j - 1,k,l + 1}} - {v_{i + 1,j + 1,k,l - 1}} + {v_{i + 1,j - 1,k,l - 1}}}}{{4\Delta {Z_1}\Delta {Z_3}}}\\
\frac{{{\partial ^2}v}}{{\partial {Z_2}\partial {Z_3}}} &= \frac{{\frac{{{v_{i + 1,j,k + 1,l + 1}} - {v_{i + 1,j,k - 1,l + 1}}}}{{2\Delta {Z_2}}} - \frac{{{v_{i + 1,j,k + 1,l - 1}} - {v_{i + 1,j,k - 1,l - 1}}}}{{2\Delta {Z_2}}}}}{{2\Delta {Z_3}}} \\
&= \frac{{{v_{i + 1,j,k + 1,l + 1}} - {v_{i + 1,j,k - 1,l + 1}} - {v_{i + 1,j,k + 1,l - 1}} + {v_{i + 1,j,k - 1,l - 1}}}}{{4\Delta {Z_2}\Delta {Z_3}}}
\end{align}

substituting the above to Equation~\ref{eqn:3varPDE} we can get the convolution equation:
\begin{align}
(r\Delta t + 1){v_{i,j,k,l}} &= \\
&(1 - \frac{{\Delta t}}{{\Delta Z_1^2}}\sigma _1^2 - \frac{{\Delta t}}{{\Delta Z_2^2}}\sigma _2^2 - \frac{{\Delta t}}{{\Delta Z_3^2}}\sigma _3^2){v_{i + 1,j,k,l}} + \\
&(\frac{{\Delta t}}{{2\Delta {Z_1}}}(r - \frac{{\sigma _1^2}}{2}) + \frac{{\Delta t}}{{2\Delta Z_1^2}}\sigma _1^2){v_{i + 1,j + 1,k,l}} + \\
&( - \frac{{\Delta t}}{{2\Delta {Z_1}}}(r - \frac{{\sigma _1^2}}{2}) + \frac{{\Delta t}}{{2\Delta Z_1^2}}\sigma _1^2){v_{i + 1,j - 1,k,l}} + \\
&(\frac{{\Delta t}}{{2\Delta {Z_2}}}(r - \frac{{\sigma _2^2}}{2}) +      \frac{{\Delta t}}{{2\Delta Z_2^2}}\sigma _2^2){v_{i + 1,j,k + 1,l}} + \\
&( - \frac{{\Delta t}}{{2\Delta {Z_2}}}(r - \frac{{\sigma _2^2}}{2}) + \frac{{\Delta t}}{{2\Delta Z_2^2}}\sigma _2^2){v_{i + 1,j,k - 1,l}} + \\
&(\frac{{\Delta t}}{{2\Delta {Z_3}}}(r - \frac{{\sigma _3^2}}{2}) + \frac{{\Delta t}}{{2\Delta Z_3^2}}\sigma _3^2){v_{i + 1,j,k,l + 1}} + \\
&( - \frac{{\Delta t}}{{2\Delta {Z_3}}}(r - \frac{{\sigma _3^2}}{2}) + \frac{{\Delta t}}{{2\Delta Z_3^2}}\sigma _3^2){v_{i + 1,j,k,l - 1}} + \\
&*\frac{{\Delta t}}{{8\Delta {Z_1}\Delta {Z_2}}}{\rho _{1,2}}{\sigma _1}{\sigma _2}({v_{i + 1,j + 1,k + 1,l}} - {v_{i + 1,j - 1,k + 1,l}} - {v_{i + 1,j + 1,k - 1,l}} + {v_{i + 1,j - 1,k - 1,l}}) + \\
&\frac{{\Delta t}}{{8\Delta {Z_1}\Delta {Z_3}}}{\rho _{1,3}}{\sigma _1}{\sigma _3}({v_{i + 1,j + 1,k,l + 1}} - {v_{i + 1,j - 1,k,l + 1}} - {v_{i + 1,j + 1,k,l - 1}} + {v_{i + 1,j - 1,k,l - 1}}) + \\
&\frac{{\Delta t}}{{8\Delta {Z_2}\Delta {Z_3}}}{\rho _{2,3}}{\sigma _2}{\sigma _3}({v_{i + 1,j,k + 1,l + 1}} - {v_{i + 1,j,k - 1,l + 1}} - {v_{i + 1,j,k + 1,l - 1}} + {v_{i + 1,j,k - 1,l - 1}})
\end{align}

\bibliographystyle{IEEEtran}
\bibliography{bib}
\end{document} 
