\documentclass[a4paper, 11pt]{report}
\usepackage{amssymb,amsmath}
\usepackage{ifxetex,ifluatex}
\usepackage{url}
\usepackage[final]{listings}
\usepackage{verbatim}
\usepackage{fixme}

%\usepackage{caption}
%\DeclareCaptionFont{white}{\color{white}}
%\DeclareCaptionFormat{listing}{\colorbox{gray}{\parbox{\textwidth}{#1#2#3}}}
%\captionsetup[lstlisting]{format=listing,labelfont=white,textfont=white}

\usepackage{tikz}
\usetikzlibrary{shapes,arrows}

\ifxetex
  \usepackage{fontspec,xltxtra,xunicode}
  \defaultfontfeatures{Mapping=tex-text,Scale=MatchLowercase}
\else
  \ifluatex
    \usepackage{fontspec}

    \defaultfontfeatures{Mapping=tex-text,Scale=MatchLowercase}
  \else
    \usepackage[utf8]{inputenc}
  \fi
\fi
$if(natbib)$
\usepackage{natbib}
\bibliographystyle{plainnat}
$endif$
$if(biblatex)$
\usepackage{biblatex}
$if(biblio-files)$
\bibliography{$biblio-files$}
$endif$
$endif$
$if(lhs)$
\usepackage{listings}
\lstnewenvironment{code}{\lstset{language=Haskell,basicstyle=\small\ttfamily}}{}
$endif$
$if(verbatim-in-note)$
\usepackage{fancyvrb}
$endif$
$if(fancy-enums)$
% Redefine labelwidth for lists; otherwise, the enumerate package will cause
% markers to extend beyond the left margin.
\makeatletter\AtBeginDocument{%
  \renewcommand{\@listi}
    {\setlength{\labelwidth}{4em}}
}\makeatother
\usepackage{enumerate}
$endif$
$if(tables)$
\usepackage{ctable}
\usepackage{float} % provides the H option for float placement
$endif$
$if(url)$
\usepackage{url}
$endif$
$if(graphics)$
\usepackage{graphicx}
% We will generate all images so they have a width \maxwidth. This means
% that they will get their normal width if they fit onto the page, but
% are scaled down if they would overflow the margins.
\makeatletter
\def\maxwidth{\ifdim\Gin@nat@width>\linewidth\linewidth
\else\Gin@nat@width\fi}
\makeatother
\let\Oldincludegraphics\includegraphics
\renewcommand{\includegraphics}[1]{\Oldincludegraphics[width=\maxwidth]{#1}}
$endif$
\ifxetex
  \usepackage[setpagesize=false, % page size defined by xetex
              unicode=false, % unicode breaks when used with xetex
              xetex]{hyperref}
\else
  \usepackage[unicode=true]{hyperref}
\fi
\hypersetup{breaklinks=true, pdfborder={0 0 0}}
$if(strikeout)$
\usepackage[normalem]{ulem}
% avoid problems with \sout in headers with hyperref:
\pdfstringdefDisableCommands{\renewcommand{\sout}{}}
$endif$
$if(subscript)$
\newcommand{\textsubscr}[1]{\ensuremath{_{\scriptsize\textrm{#1}}}}
$endif$
\setlength{\parindent}{0pt}
\setlength{\parskip}{6pt plus 2pt minus 1pt}
\setlength{\emergencystretch}{3em}  % prevent overfull lines
$if(listings)$
\usepackage{listings}
$endif$

$if(verbatim-in-note)$
\VerbatimFootnotes % allows verbatim text in footnotes
$endif$
$for(header-includes)$
$header-includes$
$endfor$

\usepackage{titlesec}
\usepackage{fancyhdr}
\usepackage{layout}
\usepackage{geometry}
\geometry{
  includeheadfoot,
  margin=3.5cm
}
\pagestyle{fancy}

\newcommand{\HRule}{\rule{\linewidth}{0.2mm}}

\newcommand{\mailto}[1]{\href{mailto:#1}{\texttt{#1}}}
\titleformat{\chapter}{\normalfont\Huge\bfseries}{\thechapter}{1em}{}
\titlespacing*{\chapter}{0pt}{0pt}{10pt}

% \setcounter{secnumdepth}{0}


\tikzstyle{decision} = [diamond, draw, fill=blue!20, text width=4.5em, text badly centered, node distance=3cm, inner sep=0pt]
\tikzstyle{block} = [rectangle, draw, fill=blue!20, text width=5em, text centered, rounded corners, minimum height=4em]
\tikzstyle{line} = [draw, -latex'] \tikzstyle{cloud} = [draw, ellipse,fill=red!20, node distance=3cm, minimum height=2em]

\begin{document}

\lstdefinestyle{MaxC}{language=C++, morekeywords={s_int32, int32,
    float8_24, sin_float8_24, sout_float8_24, float8_24, s_int,
    s_float8_24, s_bool}}

\lstdefinestyle{MaxCconf}{language=C++, morekeywords={kernels, flow,
    Host, Memory} }

\lstset{
  captionpos=b, basicstyle=\ttfamily,
  tabsize=2, numberstyle=\tiny, numbersep=10pt, breaklines=true,
  numbers=left, stepnumber=1, frame=leftline, numberblanklines=false,
  showstringspaces=false, basicstyle=\footnotesize, emph={label},
}

\begin{titlepage}

\begin{center}

{\Large Department of Computing, Imperial College London}
\HRule \\[0.4cm]
{\Large MEng Individual Project} \\ [5cm]

{\huge \bfseries Aspect Oriented Design for \\[0.25cm]Dataflow Engines}\\[0.5cm]
{\Large June 2013} \\ [3cm]

\begin{minipage}[t]{0.4\textwidth}
\begin{flushleft} \large
\emph{Author} \\[0.5cm]
{Paul Grigora\c{s}} \\[0.5cm]
%{\mailto pg1709@doc.ic.ac.uk}

\end{flushleft}
\end{minipage}
\begin{minipage}[t]{0.4\textwidth}
\begin{flushright} \large
\emph{Supervisors} \\[0.5cm]
Prof. Wayne Luk \\[0.3cm]
Dr. Stephen Weston
\end{flushright}
\end{minipage}

\vfill

\begin{minipage}{0.75\textwidth}
\begin{center}
Submitted in part fulfilment of the requirements for the degree of
Master of Engineering in Computing of Imperial College London
\end{center}
\end{minipage}

\end{center}
\end{titlepage}


$if(title)$
\maketitle
$endif$

$for(include-before)$
$include-before$

$endfor$
$if(toc)$
\tableofcontents

$endif$
$body$

$if(natbib)$
$if(biblio-files)$
$if(biblio-title)$
$if(book-class)$
\renewcommand\bibname{$biblio-title$}
$else$
\renewcommand\refname{$biblio-title$}
$endif$
$endif$
\bibliography{$biblio-files$}

$endif$
$endif$
$if(biblatex)$
\printbibliography$if(biblio-title)$[title=$biblio-title$]$endif$

$endif$
$for(include-after)$
$include-after$

$endfor$
\end{document}
