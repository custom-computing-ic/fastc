\chapter{Implementation}
\label{sec:implementation}

\section{Tools}

An initial analysis on existing compiler frameworks was carried
out. Table \ref{table:compiler-comparison} summarises the main
advantages and disadvantages of existing compiler frameworks that we
analysed for the purpose of implementing a prototype of the \FAST{}
language

\begin{table}[ht!]
\begin{tabularx}{\textwidth}{X|X|X|X|X}
  Framework & AST Manipulation & Source to Source transformation & Front-end & Supported Languages \\
  LLVM\cite{LLVM} & AST Manipulation & Source to Source transformation & Front-end & Supported Languages \\
  Cetus\cite{Cetus} & AST Manipulation & Source to Source transformation & Front-end & Supported Languages \\
  ROSE\cite{}& AST Manipulation & Source to Source transformation & Front-end & Supported Languages \\
\end{tabularx}\cite{ROSE, Quinlan:2000}
\caption{Feature comparison of compiler frameworks considered for implementation of \texttt{fastc}}
\label{table:compiler-comparison}
\end{table}

Based on these features we selected the ROSE compiler framework for the following reasons:
\begin{itemize}
\item Reliable full featured front-end support
\end{itemize}


\section{Architecture}

The compiler runs several passes on the input files:

\begin{enumerate}
\item Extract dataflow kernels
\item Infer input and output types for dataflow kernels
\item Generate MaxJ Design for extracted dataflow kernels
\item Process pragmas inside regular C code to extract a manager design
\item Process pragams inside regular C code to generate runtime code for uploading the bitstream and running the design
\end{enumerate}

To infer kernel inputs and outputs:
\begin{itemize}
\item extract kernel parameters
\item pointer parameters are streams
\item do a written analysis and record variables that are written to in the written set
\item the intersection of written and params set is the set of output streams
\item all other streams are input streams
\item only one assignment is allowed to output streams
\item transform assignments to output streams (replace them with output node in the resulting dataflow design)
\end{itemize}

It is required to infer data types as otherwise the resulting kernel
may not compile.  This task is simplified by the fact that MaxCompiler
supports some degree of type inference where the type of a variable is
inferred where possible starting from it's initial type.

A complication arises when using Boolean variables inside the
kernel. These are commonly used for enable signals and thus are
required to be supported by our approach. The problem is that C99 has
no type for Booleans just int. This makes it impossible to infer the
type Boolean for a signal whose C type is int.

We present three possiblities for addressing these issues:
\begin{itemize}
\item Adding a Boolean type to the FAST language; in the context of
  our prototype which relies heavily on compatibility with C99, this
  can be achieved by adding a macro or typedef which can be adequately
  handled by \fastc{}, directly indicating that the type of a kernel
  parameter is Boolean; an advantage of this approach is the simple
  implementation, while a potential disadvantage is the addition of
  non-standard types; this definition may  conflict with existing definitions
  in custom applications (affecting our Integration goal);

\item Use the definition of bool in the <stdbool.h> header; this may
  affect portability to Windows platforms (which do not normally use
  this header by default);

\item the alternative is to automatically infer the Boolean type based
  on the dataflow kernel; this can be achieved by an analysis of the
  function calls; thus the type of an input parameter should be
  Boolean if and only if:
  \begin{itemize}
  \item it is used as the enable signal of a hardware construct:
    counter, ROM or RAM element or DRAM command generator;
  \item its value may propagate to such a call through assignment,
    Boolean operations etc.
  \end{itemize}
  This shows that a backward dataflow analysis \cite{aho1977principles}\cite{cooper2011engineering}can be used to
  detect the set of all Boolean variables (enable signals) required
  for the kernel  \XXX{}
\end{itemize}


\section{Testing}

\section{Summary}