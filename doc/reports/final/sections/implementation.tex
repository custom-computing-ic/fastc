\chapter{Implementation}
\label{sec:implementation}

\section{Tools}

An initial analysis on existing compiler frameworks was carried
out. Table \ref{table:compiler-comparison}.

\begin{table}[ht!]
\begin{tabularx}{\textwidth}{X|X|X|X|X}
  Framework & AST Manipulation & Source to Source transformation & Front-end & Supported Languages \\
\end{tabularx}
\caption{Feature comparison of compiler frameworks considered for implementation of \texttt{fastc}}
\label{table:compiler-comparison}
\end{table}

Based on these features we selected the ROSE compiler framework for the following reasons:
\begin{itemize}
\item Reliable full featured front-end support
\end{itemize}


\section{Architecture}

The compiler runs several passes on the input files:

\begin{enumerate}
\item Extract dataflow kernels
\item Infer input and output types for dataflow kernels
\item Generate MaxJ Design for extracted dataflow kernels
\item Process pragmas inside regular C code to extract a manager design
\item Process pragams inside regular C code to generate runtime code for uploading the bitstream and running the design
\end{enumerate}

To infer kernel inputs and outputs:
\begin{itemize}
\item extract kernel parameters
\item pointer parameters are streams
\item do a written analysis and record variables that are written to in the written set
\item the intersection of written and params set is the set of output streams
\item all other streams are input streams
\item only one assignment is allowed to output streams
\item transform assignments to output streams (replace them with output node in the resulting dataflow design)
\end{itemize}
\section{Testing}

\section{Summary}