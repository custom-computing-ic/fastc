\chapter{Conclusion}

We have tackled the challenge of improving developer productivity with
minimal impact on efficiency for custom dataflow designs implemented
on FPGAs. We have shown that this can be achieved by adopting the
Aspect-oriented design philosophy of encapsulating cross-cutting
concerns (such as optimisations) in highly cohesive aspect
descriptions.

\section{Summary of Achievements}
We introduced a novel development approach for dataflow designs,
required to integrate, for the first time the Aspect-oriented approach
with dataflow design development for FPGAs. By decoupling
optimisations from design specification the proposed design flow both
simplifies the development and maintenance of dataflow applications
and highlights opportunities for platform specific optimisations.

To support the proposed design flow we introduced \FAST{} a novel
language for specifying dataflow designs which facilitates integration
with existing aspect weaving tools by adopting a standard C99
syntax. Other important features of \FAST{} include support for
hardware/software co-design, which allows embedding of dataflow
kernels in regular C style applications and support for variable bit
width operand representation via a pragma based mechanism. We
implemented a compiler for the \FAST{} language that translates
\FAST{} dataflow designs to MaxCompiler 2012.1 designs based on the
MaxCompilerRT interface. To complement the \FAST{} dataflow designs we
introduced a number of novel aspect descriptions that enable effective
design space exploration with minimal user input.

We evaluated our approach on a number of applications and showed that
significant improvements can be achieved in terms of productivity at
minimal cost to performance. We have shown \FAST{} dataflow designs
require significantly less API calls and lines of code, while matching
the performance of manually created MaxCompiler designs and improving
flexibility which simplifies the design space exploration process. Th
proposed flow can be used to support design space exploration that
highlights interesting trade-off opportunities for increasing design
parallelism or overall throughput subject to required accuracy.

A full paper based on this project was accepted for publication at the
24th IEEE International Conference on Application-specific Systems,
Architectures and Processors, ASAP 2013. Finally, the project has been
included in the FP7\footnote{European Union Seventh Framework
  Programme} funded HARNESS Project where it is to be used for
generating efficient dataflow implementations for key cloud
applications based on user requirements.

\section{Future Work}

Although we have met our original objectives set out in
\Cref{sec:intro-challenges}, we would like to highlight that
substantial work remains to be done both to improve the quality of
existing work (by improving for example error handling, documentation
and extensibility of our compiler prototype) and to investigate the
applicability of our approach to exciting classes of problems. Current
and future work possibilities include:

\begin{itemize}
\item \emph{Extending the approach to cover other classes of parallel
    computation}.  Based on the positive evaluation results, we
  believe that the approach can be extended to support the generation
  of efficient dataflow designs for additional classes of parallel
  computation such as Sparse or Dense Linear Algebra, MapReduce or
  N-Body Simulation which are quintessential examples of parallel
  kernels
  \cite{Asanovic:Bodik:Catanzaro:Gebis:Husbands:Keutzer:Patterson:Plishker:Shalf:Williams:Yelick:2006}. This
  could also provide valuable feedback which can be used to refine and
  validate the proposed approach.


\item \emph{Extension to cover heterogeneous systems}. With the
  increasing demand in cloud computing solutions, it is believed that
  heterogeneous computing platforms can provide a better mix of
  performance, energy and cost efficiency than traditional CPU only
  based platforms. From the onset of this project we have kept the
  approach as platform agnostic as possible, to enable the support for
  reconfiguration our approach is intentionally platform agnostic to
  allow extension to other platforms for dataflow computing (such as
  CPU and GPGPUs).

\item \emph{Support a standardised set of pragams}. In the context of
  heterogeneous computing platforms, to facilitate the adoption of the
  \FAST{} language a standardised set of pragmas such as OpenACC could
  be supported. We have not started with this idea in order to
  maximise the flexibility of our approach, but we believe that a
  standardised set of pragmas can help to improve portability and
  interoperability of C + \FAST{} applications.

\item \emph{Support MaxCompiler 2013.1}. MaxCompiler 2013.1 introduces
  a new interface and interesting opportunities for optimisations,
  which are highly relevant to the cloud model of shared compute
  resources. For example, it introduces the possibility to control
  groups of dataflow engines and it provides improved support for
  remote operation via Remote DMA over high speed Ininiband
  connection. These features make it interesting, if not mandatory, to
  support the more recent version of MaxCompiler.


\item The proposed design flow can be extended \emph{to support
    additional languages}. This does require a better separation in
  the \fastc{} between the dataflow specific components, which would
  serve as common backend representation for interfacing with
  MaxCompiler and the language specific front-end passes.

\item \emph{Extension to cover other applications domains}.  Based on
  the previous extension, domain specific languages for application
  domains ranging from Monte-Carlo simulations in
  finance~\cite{Jin:2012} to genetic sequence
  matching~\cite{Arram:2013} could be supported.

\item Current \emph{support for run-time facilities} is limited and more work
  is required to implement the run-time inter-facing between the CPU
  application and the dataflow designs. One of the reasons for
  postponing the implementation of run-time support was the imminent
  move to the MaxCompiler 2013.1 interface, which as mentioned
  previously introduces a heavily revised CPU -- DFE interface.

\item An interesting side-effect of our approach is that maintaining
  strict compatibility with the C99 syntax simplifies
  \emph{translation from regular C / C++} applications to \FAST{}
  dataflow designs. Hence \FAST{} itself could be used as an
  intermediate language, a target for the translation process.

\item Although based on theoretical improvements and results observed
  from our evaluation suite, we have all the reasons to believe that
  our approach can improve developer productivity and portability of
  dataflow designs a thorough study should must be carried out to
  \emph{validate our claim of improved productivity and
    portability}. This was beyond the scope of this project since it
  required a level of effort which could not have been achieved within the
  limited time span;

\item Aspect descriptions can be used to \emph{support verification}
  of designs, thus simplifying what is often a more elaborate process
  than design development itself. For example, the proposed approach
  is compatible with verification by symbolic simulation and
  equivalence checking as explained in \cite{todman2012verification};

\item Finally, the implementation presented in this report is a
  prototype and, although functional, it requires \emph{further
    engineering} work before being available for release to the
  public. In particular, error handling is not very robust and the
  extensibility of the compiler could be improved, to allow open
  programmatic access to the dataflow graph representation generated
  by the \fastc{} compiler.

\end{itemize}
