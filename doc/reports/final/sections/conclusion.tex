\chapter{Conclusion}

We introduced a novel development approach for dataflow designs. By
decoupling optimisations from design specification the proposed design
flow both simplifies the development and maintenance of dataflow
applications and highlights opportunities for platform specific
optimisations. We present extensions to our design flow that are
required to support modelling of design performance and generation of
run-time reconfigurable designs.

To support the proposed design flow we introduced \FAST{} a novel
language for specifying dataflow designs which facilitates integration
with existing aspect weaving tools by adopting a standard C99
syntax. Other important features of \FAST{} include support for
hardware/software co-design, which allows embedding of dataflow
kernels in regular C style applications and support for variable bit
width operand representation via a pragma based mechanism. We
implemented a compiler for the \FAST{} language that translates
\FAST{} dataflow designs to MaxCompiler 2012.1 designs based on the
MaxCompilerRT interface.

To complement the \FAST{} dataflow designs we introduced a number of
novel aspect descriptions that enable effective design space
exploration with minimal user input.

We evaluate our approach on a number of applications and show that
significant improvements can be achieved in terms of productivity at
minimal cost to performance. The proposed flow can be used to support
design space exploration that highlights interesting trade-off
opportunities for increasing design parallelism at minimal cost to
accuracy.

Finally, as part of the project, a full paper was accepted for
publication at the 24th IEEE International Conference on
Application-specific Systems, Architectures and Processors, ASAP 2013
and the project has been included in the FP7\footnote{European Union
  Seventh Framework Programme} funded HARNESS Project where it is to
used for generating efficient dataflow implementations for key cloud
applications based on user requirements.

\section{Future Work}


Current and future work possibilities include:

\begin{itemize}
\item \emph{Extending the approach to other classes of parallel
    computation}.  Based on the evaluation results, the approach can
  be extended to support the generation of aspect descriptions for
  additional classes of parallel computation such as Sparse or Dense
  Linear Algebra, MapReduce or N-Body Simulation which are
  quintessential examples of parallel kernels
  \cite{Asanovic:Bodik:Catanzaro:Gebis:Husbands:Keutzer:Patterson:Plishker:Shalf:Williams:Yelick:2006}. This
  could also provide valuable feedback which can be used to refine and
  validate the proposed approach.

\item \emph{Support MaxCompiler 2013.1}. MaxCompiler 2013.1 introduces
  a new interface and interesting opportunities for optimisations. It
  introduces the possibility to control groups of DFEs and.

\item \emph{Support a standardise set of pragams} such as OpenACC.

\item \emph{Extension to heterogeneous systems}. Apart from including
  specific support for reconfiguration our approach is intentionally
  platform agnostic to allow extension to other platforms for dataflow
  computing (such as CPU and GPGPUs).

\item \emph{Extension to applications in other areas of finance}
  systematic development of aspect descriptions for optimising a
  variety of application domains, from Monte-Carlo simulations in
  finance~\cite{Jin:2012} to genetic sequence
  matching~\cite{Arram:2013}.

\end{itemize}