\chapter{Conclusion}

\section{Results}

\section{Limitations}

\section{Future Work}

\emph{Extending the approach to other classes of parallel
  computation}.  Based on the evaluation results, I will investigate
extending the approach to support the generation of aspect
descriptions for additional classes of parallel computation such as
Sparse or Dense Linear Algebra, MapReduce or N-Body Simulation which
are quintessential examples of parallel kernels
\cite{Asanovic:Bodik:Catanzaro:Gebis:Husbands:Keutzer:Patterson:Plishker:Shalf:Williams:Yelick:2006}. This
step could also provide valuable feedback which can be used to refine
and validate the proposed approach.

\emph{Support MaxCompiler 2013.1}. MaxCompiler 2013.1 introduces a new
interface and interesting opportunities for optimisations. It
introduces the possibility to control groups of DFEs and...
\XXX{More details here}

\emph{Support a standardise set of pragams} such as OpenACC.

\emph{Extension to heterogeneous systems}. Apart from including
specific support for reconfiguration our approach is intentionally
platform agnostic to allow extension to other platforms for dataflow
computing (such as CPU and GPGPUs).

\emph{Extension to applications in other areas of finance}
