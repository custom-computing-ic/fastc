\chapter{Original RTM Kernel}
\label{app:rtm-kernel}
The following listing shows the original compute kernel for the
Reverse Time Migration application implemented with MaxJ using
MaxCompiler 2012.1. Some typical Java constructs such as package
definitions and imports are omitted.

\begin{lstlisting}[
  basicstyle=\footnotesize\sffamily,
  keywordstyle=\color{black}\bfseries,
  numbers=left,
  captionpos=b,
  breaklines=true,
  frame=l,
  xleftmargin=2em,
  escapeinside={(*@}{@*)}
]
public class RTM extends Kernel {
  int Par=1, Mul=1, Sub=0; (*@\label{app:rtmk-par}@*)

  public KArrayType<HWVar> burst_in=
    new KArrayType<HWVar>(hwFloat(8, 24),  Par);

  public KArrayType<HWVar> burst_out=
    new KArrayType<HWVar>(hwFloat(8, 24),  Par);

  public RTM(KernelParameters parameters) {
    super(parameters);
    HWFloat real    = hwFloat(8,24);
    HWFix   fix_4_24= hwFix(4,24,HWFix.SignMode.TWOSCOMPLEMENT);

    OffsetExpr nx = stream.makeOffsetParam("nx",  24/Par, 48/Par);
    OffsetExpr nxy = stream.makeOffsetParam("nxy",32* nx, 32 * nx);

    // Application Specific constants omitted
    (...)
    HWVar bc    = constant.var(-0.0005);

    HWVar n1    = io.scalarInput("n1",    hwUInt(32));
    HWVar n2    = io.scalarInput("n2",    hwUInt(32));
    HWVar n3    = io.scalarInput("n3",    hwUInt(32));
    HWVar ORDER = io.scalarInput("ORDER", hwUInt(32));
    HWVar SPONGE= io.scalarInput("SPONGE",hwUInt(32));

    CounterChain chain = control.count.makeCounterChain();
    HWVar i4 = chain.addCounter(1000,1).cast(hwUInt(32));//iteration
    HWVar i3 = chain.addCounter(n3,  1).cast(hwUInt(32));//outest loop
    HWVar i2 = chain.addCounter(n2,  1).cast(hwUInt(32));
    HWVar i1 = chain.addCounter(n1,Par).cast(hwUInt(32));//innest loop

    HWVar up[] = new HWVar[Par];
    for (int i=0; i <Par; i++)
    up[i] = i3>=ORDER & i3<n3-ORDER  & i2>=ORDER & i2<n2-ORDER  & i1>=ORDER-i  & i1<n1-ORDER-i;

    HWVar output_ring = n3  > i3  & i3 >= n3 - ORDER;

    HWVar input_ring = ORDER > i3;

    HWVar output_enable = (n3 - ORDER > i3 & i3 >= ORDER);

    // input
    KArray<HWVar> burst_p     =io.input("burst_p",      burst_in);
    KArray<HWVar> burst_pp    =io.input("burst_pp",     burst_in);
    KArray<HWVar> burst_dvv   =io.input("burst_dvv",    burst_in);
    KArray<HWVar> burst_source=io.input("burst_source", burst_in);

    HWVar p[]       =new HWVar[Par];
    HWVar pp_i[]    =new HWVar[Par];
    HWVar dvv[]     =new HWVar[Par];
    HWVar source[]  =new HWVar[Par];

    HWVar image[][] =new HWVar[Par][Mul];

    for (int i=0; i <Par; i++)
    {
      p[i]=       burst_p[i].cast(real);
      pp_i[i]=    burst_pp[i].cast(real);
      dvv[i]=     burst_dvv[i].cast(real);
      source[i]=  burst_source[i].cast(real);
    }

    HWVar cur[][][][]    = new HWVar[Mul][11+Par+1][11][11];
    HWVar inter[][]      = new HWVar[Par][Mul];
    HWVar result[][]     = new HWVar[Par][Mul];


    optimization.pushDSPFactor(1);
    //Cache
    for (int i=0; i <Par; i++)
    {
      int k = -6/Par;
      for (int x=-6; x<=6; x+=Par)
      {
        for (int y=-5; y<=5; y++)
          for (int z=-5; z<=5; z++)
            cur[0][x+6+i][y+5][z+5] = stream.offset(p[i], z*nxy+y*nx+k);
        k++;
      }
    }
    //Computation
    for (int i=0; i <Par; i++)
    {
    //data-path(0,i)
    result[i][0]=( (*@\label{app:rtmk-op1}@*)
                    cur[0][6+i][5][5] * 2.0 - pp_i[i] +dvv[i]*(
                    cur[0][6+i][5][5] * c_0
                  +(cur[0][5+i][5][5] + cur[0][7+i][5][5]) * c_1_0
                  +(cur[0][4+i][5][5] + cur[0][8+i][5][5]) * c_1_1
                  +(cur[0][3+i][5][5] + cur[0][9+i][5][5]) * c_1_2
                  +(cur[0][2+i][5][5] + cur[0][10+i][5][5])* c_1_3
                  +(cur[0][1+i][5][5] + cur[0][11+i][5][5])* c_1_4
                  +(cur[0][6+i][4][5] + cur[0][6+i][6][5]) * c_2_0
                  +(cur[0][6+i][3][5] + cur[0][6+i][7][5]) * c_2_1
                  +(cur[0][6+i][2][5] + cur[0][6+i][8][5]) * c_2_2
                  +(cur[0][6+i][1][5] + cur[0][6+i][9][5]) * c_2_3
                  +(cur[0][6+i][0][5] + cur[0][6+i][10][5])* c_2_4
                  +(cur[0][6+i][5][4] + cur[0][6+i][5][6]) * c_3_0
                  +(cur[0][6+i][5][3] + cur[0][6+i][5][7]) * c_3_1
                  +(cur[0][6+i][5][2] + cur[0][6+i][5][8]) * c_3_2
                  +(cur[0][6+i][5][1] + cur[0][6+i][5][9]) * c_3_3
                  +(cur[0][6+i][5][0] + cur[0][6+i][5][10])* c_3_4 ))
                  + source[i];
    inter[i][0]   =(up[i])? result[i][0] : pp_i[i];
    }
    //Multiple Time-Dimension
    for (int j=1; j <Mul; j++)
    {
      //Cache
      for (int i=0; i <Par; i++)
      {
        int k = -6/Par;
        for (int x=-6; x<=6; x+=Par)
        {
          for (int y=-5; y<=5; y++)
            for (int z=-5; z<=5; z++)
              cur[j][x+6+i][y+5][z+5] = stream.offset(inter[i][j-1], z*nxy+y*nx+k);
          k++;
        }
      }

      //Computation
      for (int i=0; i <Par; i++)
      {
      //data-path(j,i)
      result[i][j]=( (*@\label{app:rtmk-op2}@*)
                      cur[j][6+i][5][5] * 2.0 - cur[j-1][6+i][5][5] +dvv[i]*(
                      cur[j][6+i][5][5] * c_0
                    +(cur[j][5+i][5][5] + cur[j][7+i][5][5]) * c_1_0
                    +(cur[j][4+i][5][5] + cur[j][8+i][5][5]) * c_1_1
                    +(cur[j][3+i][5][5] + cur[j][9+i][5][5]) * c_1_2
                    +(cur[j][2+i][5][5] + cur[j][10+i][5][5]) * c_1_3
                    +(cur[j][1+i][5][5] + cur[j][11+i][5][5])* c_1_4
                    +(cur[j][6+i][4][5] + cur[j][6+i][6][5]) * c_2_0
                    +(cur[j][6+i][3][5] + cur[j][6+i][7][5]) * c_2_1
                    +(cur[j][6+i][2][5] + cur[j][6+i][8][5]) * c_2_2
                    +(cur[j][6+i][1][5] + cur[j][6+i][9][5]) * c_2_3
                    +(cur[j][6+i][0][5] + cur[j][6+i][10][5])* c_2_4
                    +(cur[j][6+i][5][4] + cur[j][6+i][5][6]) * c_3_0
                    +(cur[j][6+i][5][3] + cur[j][6+i][5][7]) * c_3_1
                    +(cur[j][6+i][5][2] + cur[j][6+i][5][8]) * c_3_2
                    +(cur[j][6+i][5][1] + cur[j][6+i][5][9]) * c_3_3
                    +(cur[j][6+i][5][0] + cur[j][6+i][5][10])* c_3_4 ))
                    + source[i];
      inter[i][j]   =(up[i])? result[i][j] : cur[j-1][6+i][5][5];
      }
    }

    //setup configuration
    optimization.popDSPFactor();

    // control counter
    KArray<HWVar> output_p  = burst_out.newInstance(this);
    KArray<HWVar> output_pp = burst_out.newInstance(this);

    for (int i=0; i <Par; i++)
    {
      output_p[i]  <==       inter[i][Mul-1].cast(hwFloat(8,24));
      output_pp[i] <== cur[Mul-1][6+i][5][5].cast(hwFloat(8,24));
    }

    io.output("ker_p",     output_p,  burst_out);
    io.output("output_pp", output_pp, burst_out);

  }
}
\end{lstlisting}