\section{Design Flow}
\label{sec:design-flow}

To improve both efficiency and productivity, our design flow focuses
on maintaining or improving the \emph{performance} and \emph{energy
  efficiency} of existing applications while using a more systematic
approach for design optimisation that results in more \emph{portable}
application code, improves \emph{integration} with existing
applications and \emph{automates} manual, time consuming and
error-prone tasks.

\begin{comment}
Our design flow aims to improve both \emph{efficiency} (in terms of
performance and energy consumption) and \emph{productivity}. The
former is crucial to High Performance Computing, the latter helps
reduce development cost and time and is a well-known issue with
existing FPGA based acceleration solutions \cite{jones2010gpu}. To
achieve this we focus on maintaining or improving the
\emph{performance} and \emph{energy efficiency} of existing
applications while using a more systematic approach for design
optimisation that results in more \emph{portable} application code,
improves \emph{integration} with existing applications and
\emph{automates} manual, time consuming and error-prone tasks.
\end{comment}

The key components of the design flow are:
\begin{itemize}
\item \emph{\FAST{}}, a novel language for specifying dataflow designs
  which is compatible with C syntax, improving developer productivity
  and supporting combined hardware and software specifications;

\item \emph{aspect driven compilation flow}, used to decouple
  optimisation from design development, improving design portability,
  and automating the generation of code and design space
  exploration~\footnote{the process of exploring multiple
    configurations to identify optimal implementations} which improves
  productivity;

\item \emph{systematic design space exploration}, to identify maximum
  performance configurations by using \emph{aspect descriptions} to
  conveniently control and guide the exploration process based on
  user-specified constraints;
\end{itemize}

\Cref{fig:design-flow} illustrates the design flow:
\begin{enumerate}
\item a C application containing an embedded high-level dataflow
  design specified in \FAST{} is developed from the original source
  application;
\item the dataflow design is transformed by the aspects in the
  repository to generate new configurations (e.g. with multiple
  word-length configurations);
\item the generated configurations are compiled using a backend
  compilation toolchain (MaxCompiler) to dataflow designs
  implemented on FPGAs;
\item the feedback from the compilation process is used to drive
  design space exploration, repeating the weaving and compilation
  process until user-specified constraints are satisfied.
\end{enumerate}

\begin{figure}[!ht]
  \centering
  \def\svgwidth{0.8\linewidth}
  \input{figs/asap13-design-flow.pdf_tex}
  \caption{Proposed approach for aspect-driven compilation of dataflow
   designs.}
  \label{fig:design-flow}
\end{figure}