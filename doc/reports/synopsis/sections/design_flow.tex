\section{Design Flow}
\label{sec:design-flow}

To improve both efficiency and productivity, our design flow improves
the \emph{performance} and \emph{energy efficiency} of existing
applications while using a more systematic approach for design
optimisation that results in more \emph{portable} application code,
improves \emph{integration} with existing applications and
\emph{automates} manual, time-consuming and error-prone tasks.

The key components of the design flow are:
\begin{itemize}
\item \emph{\FAST{}}, a novel language for specifying dataflow designs
  which is compatible with C syntax, improving developer productivity
  and supporting combined hardware and software specifications;

\item \emph{aspect driven compilation flow}, used to decouple
  optimisation from design development, improving portability, and
  automating code generation and design space
  exploration~\footnote{the process of exploring multiple
    configurations to identify optimal implementations}, improving
  productivity;

\item \emph{systematic design space exploration}, to identify
  efficient configurations by using \emph{aspect descriptions} to
  control the exploration process based on user-specified constraints.
\end{itemize}

\Cref{fig:design-flow} illustrates the design flow:
\begin{enumerate}
\item a C application containing an embedded high-level dataflow
  design specified in \FAST{} is developed from the original source
  application;
\item the dataflow design is transformed according to the aspect
  descriptions in the repository to generate new configurations
  (e.g. with multiple word-length configurations);
\item the generated configurations are compiled using a backend
  compilation toolchain to FPGA dataflow designs;
\item feedback from the compilation process is used to drive design
  space exploration until user-specified constraints are satisfied.
\end{enumerate}

\begin{figure}[!ht]
  \centering
  \def\svgwidth{0.9\textwidth}
  \input{figs/synopsis-design-flow.pdf_tex}
  \caption{Proposed approach for aspect-driven compilation of dataflow
   designs.}
  \label{fig:design-flow}
\end{figure}