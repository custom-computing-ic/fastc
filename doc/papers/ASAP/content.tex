\section{Introduction}

The dataflow computing paradigm is more suitable for implementing high
throughput, highly parallel applications that operate on large amounts
of uniform data than general purpose architectures. Existing work
shows that dataflow machines emulated on FPGAs achieve performance
gains of up to several orders of magnitude compared to traditional
control based architectures \cite{Flynn:Pell:Mencer:2012},
\cite{Mencer:2012},
\cite{Oriato:Tilbury:Marrocu:Pusceddu:2012}. Because they eliminate
the fetch-decode-execute cycle of traditional von Neumann machines
\cite{Neumann:1993}, dataflow designs require less area for caching
and control (e.g. branch prediction) which leads to the increase in
performance and decrease in power consumption. Another key aspect is
that due to its regularity, a dataflow design can be statically
scheduled into a deep hazard-free pipeline, effectively achieving the
ideal throughput rate of one result per clock cycle. This makes
dataflow designs more suitable for expressing high throughput, regular
computations than general purpose computing devices. However, dataflow
languages are not commonly used and even with the advent of more
modern functional programming languages, imperative languages such as
Java and C/C++ remain most popular \cite{Tiobe:2012}.

We identify the following challenges that should be overcome to
facilitate the adoption of dataflow designs:
\begin{enumerate}
\item Specifying dataflow designs, in an intuitive, well understood
  language that is concise and facilitates the translation of existing
  designs and expressive enough to support the requirements of modern
  high-performance applications.
\item Specifying optimization strategies, decoupled from the
  application code in a manner that makes the specification easy to
  reuse and customize and is comprehensive enough to allow capturing
  of optimizations at various levels: \emph{algorithmic
    transformations} that enable transformation of the original
  application to expose parallelism or improve communication between
  CPU and accelerator, \emph{design-level transformations} that
  enable exploration of platform specific optimizations and
  \emph{productivity related transformations} that improve developer
  productivity.
\item Systematic design space exploration of dataflow designs driven
  by these parameterizable optimization strategies, that increase
  developer productivity and allow exploration of design level
  trade-offs.
\item Applying these design techniques to create and optimize high
  performance applications from existing implementations
\end{enumerate}

We propose a methodology for addressing these challenges based on the
following contributions:
\begin{enumerate}
\item We introduce MaxC, a dataflow language based on C99 that can be
  used to create high-performance designs. We implement a compiler
  that translates MaxC designs to MaxJ an existing commercial dataflow
  language. These are then compiled using MaxCompiler 2012.1 and
  executed on a MAX3424 Board with a Virtex 6 FPGA chip and 48GB of
  on-board DRAM.
\item We introduce novel aspects for specifying optimisation
  strategies at the algorithmic level, the design level and
  productivity level. We implement these aspects using
  LARA, an aspect oriented language for FPGA systems.
\item We propose an automated method for design space exploration of
  MaxC dataflow designs, driven by the aspect definitions.
\item We evaluate our approach implementing a high performance design
  for an application based on the Reverse Time Migration technique for
  seismic imaging and comparing it with existing CPU, GPU and FPGA
  implementations.
\end{enumerate}

\section{Related Work}

\TODO Half page of related work on dataflow languages and aspect
driven design.

\subsection{Dataflow Languages}

\begin{itemize}

\item Streams-C -

\item StreamIt -, Sequoia++ - different, non-standard lanuages. [We
  found them hard to learn/understand follow.]

\item MaxJ - It’s difficult to manage the two components of the
  projects: two different languages, complicates the build system and
  teh design space exploration (since it is harder to share data
  between designs, like we can in MaxC with headers). Java is more
  verbose and in most designs this is not necessarilly useful (we may
  not benefit that much from high-level constructs) . In C we can use
  macros to capture repetitive patterns. On the other hand C is not an
  OO language, so this may be bad for code reuse.

\item ImpulseC - commercial version of Streams-C

\end{itemize}

\subsection{Aspect Driven Design}


\section{The  MaxC Language}

MaxC is a novel language for specifying dataflow designs. It is based
on the widely used C99 standard which makes it familiar and easy to
adopt for developers facilitating translation of existing application
code to dataflow designs. The simple syntax of C99 facilitates
integration with other tools (such as aspect weavers), allowing the
language to interact well with existing compilers or source to source
translation frameworks (e.g. LARA, ROSE), allowing source level
optimizations to be applied through different tools. MaxC itself is a
simple language, which is intended to be used for expressing the
simplest form of a dataflow design. Optimizations and other
transformations are encapsulated in aspects which are developed
separately and applied through aspect weaving. This results in a more
flexible approach for generating and exploring the space of efficient
dataflow designs.

\begin{comment}
We identify the following requirements for any dataflow language:
\begin{enumerate}
\item intuitive and easy to use
\item facilitate translation of existing applications
\item interacts well with
high-level tools
\end{enumerate}
\end{comment}

Since MaxC is compiled to MaxJ it uses the same execution model: a
design is composed of one or more computational kernels (which
implement the functionality we are interested in) which are connected
to form a design. Communication between kernels is asynchronous, so
kernels can operate independently of each other but computation within
kernels is synchronous, so a kernel will compute only when all it's
active inputs have values available.

Figure \ref{fig:maxc-1dconv} shows an example MaxC dataflow kernel
which implements a 1D convolution computation used to value European
options.


\lstset{style=MaxC}

\begin{figure}
\begin{lstlisting}
void kernel_Convolution1D(float* p, float c_0_0_0, float c_p_0_0, float c_n_0_0, int n1, int ORDER, float* out)
{

  in(p);

  float* i4 = count(1000, 1);
  float* i1 = countChain(n1, 1, i4);

  #pragma DSPBalance:full
  int result =
    data[0]  * c_0_0_0 +
    data[1]  * c_p_0_0 +
    data[-1] * c_n_0_0;

  int up = (i1 >= ORDER) && (i1 < n1 - ORDER);

  out = stream_select(up, data, result);

  out(out);
}
\end{lstlisting}
\caption{Example MaxC design for a 1D convolution kernel.}
\label{fig:maxc-1dconv}
\end{figure}

\subsection{Kernels}

Kernels are defined as regular C functions, with the ``kernel\_''
identifier prepended (Line 1). The inputs and outputs of the kernel
are clearly specified in the header, inputs followed by outputs. At
the kernel level MaxC captures the dataflow elements by using:

\begin{itemize}
\item \emph{regular C constructs} are used as much as possible to make
  the language more intuitive. This includes standard C99 types and
  conventions for implementing kernels. Some overloaded operators are
  provided such as the array access operators to provide a more
  succinct syntax for accessing ``past'' or ``future'' stream elements
  (Lines 15 -- 17).

\item \emph{pragma directives} are used to convey additional
  information to the compiler for which C99 does not provide flexible
  built-in constructs (e.g. specifying type width) or additional
  optimization hints (e.g. the DSP balance, Line 13);

\item \emph{specific API calls} are used for higher level constructs
  such as inputs and outputs (Lines 8, 24), counters (Lines 10 -- 11).

\end{itemize}


Figure \ref{fig:maxc-1dconv} shows the main elements of a MaxC design:

\begin{enumerate}
\item \emph{inputs and outputs} are declared in the kernel header and
  connected in the kernel body. Lines 2-6 show the kernel declaration
  specifying the stream input "p" and output "out" as well as a number
  of scalar parameters configurable at run-time. API calls are used to
  connect these inputs (Line 7 and 22);

\item \emph{control} elements are implemented either via standard C99
  constructs (such as conditional statements or operators) or via API
  calls that enable multiplexing between 2 or more streams; API:
  fselect(cond, stream1, stream2)

\item \emph{computation} is implemented using regular C99 syntax and
  semantics. API functions are provided for hardware blocks that
  implement common functions such as square root, exponential,
  logarithms.


\item \emph{streams} are represented as regular C99 pointers. For
  example Line 2 declares a stream of float values names ``p''.
  Normal array notation can be used to generate either previous
  (negative indices) or future values (positive indices) or
  dereference the stream to get the current stream value. Negative
  indices are allowed (as on Line 16). const float c; compile time
  constant, used to parametrize designs On lines 13-16 we use array
  index notation to access future (positive offset) or past (negative
  offset) stream elements.

\item \emph{compile time constructs} such as loops are supported as
  long as their bounds are known at compile time.

\item \emph{optimizations} can be applied via pragmas (Line 12) or API
  calls. Additional type information can also be provided via pragmas.

\end{enumerate}

The MaxC API provides a number of useful, higher-level constructs.
such as output functions that are used to connect an internal kernel
stream to the output stream of a kernel, various counters and counter
chain configurations that can be instantiated using functions from the
counter API (Lines 9 -- 10) or functions such as stream\_select that
are used to multiplex between a number of streams based on the value
of condition stream.




\subsection{Designs}

%\XXX{Command read/write kernel}

%\XXX{Design diagram?}

MaxC also allows specification of designs using multiple kernels. This
involves selecting the kernel instances and connecting them as well as
setting a number of design configuration and compilation options such
as operating frequency.

Figure \ref{lst:maxc-design} illustrates the design for the 1D
convolution example which places and connects memory command kernels
to control the generation of memory streams and the actual computation
kernels.

On Lines 3-6 we specify the kernel instances. On Lines 8 -- 10 we
connect these. Lines 13 -- 15 specify the design frequency, the memory
clock frequency and enable the addition of debug elements.

\begin{figure}[!h]
\centering
\begin{lstlisting}
  void design_Convolution1D() {
    // kernels
    kernel_t k1 = kernel_init(kernel_Convolution1D);
    kernel_t k2 = kernel_init(kernel_Cmdwrite);
    kernel_t k3 = kernel_init(kernel_Cmdread);

    // connections
    connect2(k1, k2);
    connect3(k1, k2, ``a'');
    connect4(k1, k3, ``a'', ``b'');

    // configuration
    set_frequency(150);
    set_memory_frequency(333);
    set_enable_debug(true);
  }
\end{lstlisting}
\caption{MaxC design specification, connecting multiple kernels and
  setting various configuration options}
\label{lst:maxc-design}
\end{figure}

\subsection{Comparison with Other Dataflow Languages}

\TODO Identify relevant comparison features after Related Work section
is finished.

Table \ref{table:feature-comparison} summarizes the features of MaxC
in comparison with other languages described in the related works
section.

\begin{table}[!h]
  \renewcommand{\arraystretch}{1.3}
  \centering
  \caption{Feature comparison of MaxC and other dataflow languages.}
  \label{table:feature-comparison}
  \begin{tabular}{ l | c |  p{1cm} |  p{1cm} |  c |  c }
    Language  & Syntax & FPGA Support & Execution Model & Debugging & Simulation \\ \hline
    MaxJ      & Java   & Yes          & F3              & F4        & F5         \\
    Streams-C & F1     & F2           & F3              & F4        & F5         \\
    ImpulseC  & F1     & F2           & F3              & F4        & F5         \\
    StreamIt  & F1     & F2           & F3              & F4        & F5         \\
    Sequoia++ & F1     & F2           & F3              & F4        & F5         \\
    MaxC      & C99    & Yes          & F3              & F4        & F5         \\
  \end{tabular}
\end{table}

\section{Aspects}

MaxC was designed to integrate with aspect weaving tools by using
standard C99 syntax. This allows MaxC specifications to describe the
simplest form of a dataflow design while aspects specify decoupled
optimization and transformation strategies that operate on this
design. This make the functionality of the application easier to
understand, more maintainable and portable since it is no longer
obscured by various structural or algorithmic transformations or
platform specific optimizations.

We distinguish three categories of aspects: \emph{structural aspects},
\emph{design aspects} and \emph{development aspects}.

\subsection{Structural Aspects}

Structural aspects capture transformation or optimization strategies
that alter the structure or the algorithm of the original application
in order to expose parallelism or improve communication between CPU
and accelerator via hardware/software partitioning.

\subsubsection{Translation}

Translation aspects tranform high-level source code to MaxC designs.

Aspect steps:
\begin{enumerate}

\item identify acceleration candidates by profiling the
  computation. In particular analyse loops with computations and
  profile at various degrees of nesting;

\item transform local arrays to dynamic arrays;

\item extract computational kernel by mapping the outer loop to the
  stream loop;

\item generate runtime API (currently using MaxCompilerRT).

\end{enumerate}

This aspect will map the C99 application shown in Figure
\ref{fig:c-design} to Lines 12 -- 16 of the 1D Convolution kernel in
Figure \ref{fig:maxc-1dconv}.


\begin{figure}[!h]
\centering
\begin{lstlisting}
void kernel(int source[], int m, float a_0_0_0, float a_p1_0_0, float a_m1_0_0) {

  for(j=1; j<m; j++){
    target[j] =
        source[j]  * a_0_0_0
      + source[j + 1] * a_p1_0_0
      + source[j - 1] * a_m1_0_0;
    }

}
\end{lstlisting}
\caption{Original C99 source code for the 1D convolution kernel in
  Figure \ref{fig:maxc-1dconv}.}
\label{fig:c-design}
\end{figure}

\subsubsection{\TODO Reconfiguration}


\subsubsection{\TODO Parallelism}

\subsection{Design Aspects}

Design aspects capture low level design optimizations that can be
applied in order to improve timing or explore various resource usage
trade-offs.

For example aspects can be used to enable design space exploration of
resource usage vs. accuracy trade-offs by varying the word lengths.

\TODO Example?

Alternatively we can use them to explore resource trade-offs by
mapping computation to specialised FPGA blocks. For example, the
aspect in Figure \ref{fig:aspect-DSP} can be used to map arithmetic
expressions to DSP blocks. We define the DSP balance factor to be used
(none, normal, full), the granularity (which decides if we break up
arithmetic expressions to insert DSP Factors between them) and the
number of independent applications.

\lstset{style=aspectp}
\begin{figure}[!h]
\centering
\begin{lstlisting}
aspectdef OptimizeDSPUsage(
 DSPBalance, granularity, applications)
{
    where: arithmetic expression occurs
    what:
      split expression if it has more than granularity operations
      insert DSP balance pragma
}
\end{lstlisting}
\caption{Aspect for exploring mapping of computation to DSP blocks.}
\label{fig:aspect-DSP}
\end{figure}

Aspects steps:

\begin{enumerate}
\item Identify arithmetic expressions;
\item Split them according to granularity;
\item Insert pragma to adjust DSP usage;
\item Repeat for a given number of applications.
\end{enumerate}

\newsavebox{\secondlisting}
\begin{lrbox}{\secondlisting}% Store second listing
\lstinputlisting[firstline=8]{code.txt}
\end{lrbox}

\begin{figure}
\begin{lstlisting}
  #pragma balanceDSP=full
  int result = p[0]  * c_0_0_0;
  int result = result + p[1]  * c_p_0_0;
  #pragma balanceDSP=full
  int result = result + p[-1] * c_n_0_0;
\end{lstlisting}
\caption{One possible result of applying
  \texttt{OptimizeDSPUsage(full, fine, 2)} to Lines 13 -- 17 of Figure
  \ref{fig:maxc-1dconv}.}
\label{fig:maxc-1dconv-aspect}
\end{figure}

Figure \ref{fig:maxc-1dconv-aspect} shows one of the possible results
of applying the aspect \texttt{OptimizeDSPUsage(full, fine, 2 } to
Lines 13 -- 17 of Figure \ref{fig:maxc-1dconv}.

Using LARA we can implement and combine these aspects to enable
systematic design space exploration of all the optimisations options
exposed by the MaxC backend resulting in the generation of a large
number of designs. The feedback-direct compilation process of LARA can
be used to capture and extract feedback from the backend reports
pertaining to resource usage or timing information and automatically
adjust the compilation process.

\TODO include some exploration data for the 1D Convolution example;
could also use some resource annotation data to show the effect of
pushing the DSP factor

\subsection{Development Aspects}

Development aspects capture transformations that have an impact on the
development process such as debugging, simulating kernels or improving
compilation speed. Separating these concerns makes the original
application code easier to maintain and enables the automatic
application of these transformations to a wide range of designs, thus
improving developer productivity.

\subsubsection{Simulation Aspects}

The goal for the MaxC kernel simulation model is that it should be
possible to compile and run dataflow designs using the standard GCC
toolchain in order to verify the logical correctness of the
design. However, this often leads to the need of adding boilerplate
code that would otherwise not be required by the MaxC backend, just to
enable simulation. Since this process is itself manual and hence
error-prone it defeats the very purpose of testing the design. One
approach would be to require users to always use the simulation
API. However this unnecessarily complicates the dataflow design. Our
solution is to use separate aspects to generate the simulation
designs.

Simulation aspects can be applied to the original dataflow design to
enable pure software simulation. This approach gives more freedom when
designing MaxC constructs -- since we are not necessarily concerned
weather they compile using the standard GCC toolchain. This in turn
allows us to provide a neater syntax but also hides and automates the
details of generating simulation designs from the developer.

Aspect steps:

\begin{enumerate}

\item insert kernel simulation loop in hostcode %(Figure )


\begin{figure}[!h]
\begin{lstlisting}
  for (int i = 0 ; i < CYCLE_COUNT; i++) {
    kernel_Convolution1d(p, out, 2);
    update_stream_pointers(p, out);
  }
\end{lstlisting}
%\caption{Stream simulation loop.}
%\label{fig:maxc-simulation}
\end{figure}

\item insert boilerplate code in the dataflow design to enable simulation.
 % The result of applying the simulation aspect to the dataflow in
%  Figure \ref{fig:maxc-1dconv} is shown in Figure
%  \ref{fig:maxc-sim-aspect}.
  For example any stream and counter used in the kernel body need to
  be declared using special calls that also pass in unique identifiers
  which are used to store and retrieve the values at each cycle in a
  separate data structure (e.g. \texttt{`stream\_init\_i(int
    stream\_id)}). On each kernel cycle, the function then either
  allocates a new stream or increments the stream pointer if the
  stream has already been allocated.
\begin{comment}
\begin{figure}[!h]
\begin{lstlisting}
  float* i4 = count_i(1000, 1, 0);
  float* i1 = countChain_i(n1, 1, i4, 0);
\end{lstlisting}
\caption{Applying the simulation aspect to the dataflow design in
  Figure \ref{fig:maxc-1dconv}}
\label{fig:maxc-sim-aspect}
\end{figure}
\end{comment}

\end{enumerate}

\lstset{style=MaxC}

\subsubsection{Debug Aspects}

Because the current execution model does not provide for runtime
debugging of hardware designs the easiest solution to debug designs is
to log the values of various streams during execution. The insertion
of debug statements can be encapsulated in aspects. It is particularly
important to separate debug aspects from the original application code
since debug blocks can influence the compilation time and timing
constraints as well as the behaviour of the design.

\lstset{style=aspectp}
\begin{figure}[!h]
\centering
\begin{lstlisting}
aspectdef DebugValues()
{
    where: variable assignment
    what:
      log value before assignment
      log value after assignment
}
\end{lstlisting}
\caption{Aspect for exploring mapping of computation to DSP blocks.}
\label{fig:aspect-DSP}
\end{figure}


\TODO Include some data on how this influences code size,
  productivity

\subsubsection{Compilation Aspects}

Compilation aspects can be applied during the development process to
create versions of the dataflow design that compile faster. They apply
changes such as reducing the operating frequency, removing debug
blocks or applying design-level optimizations that can resolve timing
constraints. Naturally reducing the compilation time increases
developer productivity.

The various aspects can be used to build in parallel more versions of
the application that allow testing the production version (maximum
performance), a more naive version which builds faster and a debug
version which allows debugging the naive version.

\TODO Add data about how aspects can influence compilation time

\TODO Add an example aspect?


\section{Proposed Design Flow}

We perform a DSE step until we find a design that fits the chip,
achieves timing closure and meets non functional requirements.  For
each optimization strategy in our repository we:

\begin{enumerate}

\item apply the set of specific low-level optimizations comprised in
  the strategy (e.g. setting DSP balance).

\item  we compile the optimized
  design using the MaxCC backend to MaxJava, Maxeler's own dataflow
  language

\item we start the backend compilation toolchain (MaxCompiler
  and Xilinx) and perform an analysis of the reporting information,
  based on which we either restart the flow with the next optimization
  strategy step or proceed to the next step

\item we measure NFRs such as performance or latency and if our target
  is not met we restart using a different optimization strategy

\end{enumerate}

\begin{figure}
\caption{This is very long caption textasd. This is very long caption
  textasd. This is very long caption textasd. This is very long
  caption textasd.}
\begin{center}
  \begin{tikzpicture}[node distance = 2cm, auto]
    \node [block] (csrc) {C source};
    \node [block, below of=csrc, left of=csrc] (maxrt) {CPU Runtime code};
    \node [block, below of=csrc, right of=csrc] (maxc) {MaxC Design};
    \node [block, below of=maxc] (maxj) {MaxJ Design};
    \node [block, below of=maxj] (maxfile) {Maxfile};
    \node [block, below of=maxfile, left of=maxfile] (app) {Application executable};
    \node [block, below of=app] (maxnode) {MaxNode, MaxStation};

    \node [cloud, right of=csrc] (lara) {LARA};
    \node [cloud, left of=csrc] (maxcc) {MaxCC};
    % \node [cloud, right of=init] (system) {system};
    % \node [block, below of=init] (identify) {identify candidate models};
    % \node [block, below of=identify] (evaluate) {evaluate candidate models};
    % \node [block, left of=evaluate, node distance=3cm] (update) {update model};
    % \node [decision, below of=evaluate] (decide) {is best candidate better?};
    % \node [belock, below of=decide, node distance=3cm] (stop) {stop};

    \path [line] (csrc) -- (maxrt);
    \path [line] (csrc) -- (maxc);
    \path [line] (maxc) -- (maxj);
    \path [line] (maxj) -- (maxfile);
    \path [line] (maxfile) -- (app);
    \path [line] (maxrt) -- (app);
    \path [line] (app) -- (maxnode);
    \path [line, dashed] (lara) -- (csrc);
    \path [line, dashed] (lara) -- (maxc);

    % \path [line] (decide) -| node [near start] {yes} (update);
    % \path [line] (update) |- (identify);
    % \path [line] (decide) -- node {no}(stop);
    % \path [line,dashed] (expert) -- (init);
    % \path [line,dashed] (system) -- (init);
    % \path [line,dashed] (system) |- (evaluate);
  \end{tikzpicture}
\end{center}
\end{figure}
% An example of a floating figure using the graphicx package.
% Note that \label must occur AFTER (or within) \caption.
% For figures, \caption should occur after the \includegraphics.
% Note that IEEEtran v1.7 and later has special internal code that
% is designed to preserve the operation of \label within \caption
% even when the captionsoff option is in effect. However, because
% of issues like this, it may be the safest practice to put all your
% \label just after \caption rather than within \caption{}.
%
% Reminder: the "draftcls" or "draftclsnofoot", not "draft", class
% option should be used if it is desired that the figures are to be
% displayed while in draft mode.
%
% \begin{figure}[!t]
%   \centering
%   \includegraphics[width=2.5in]{myfigure}
%   where an .eps filename suffix will be assumed under latex,
%   and a .pdf suffix will be assumed for pdflatex; or what has been declared
%   via \DeclareGraphicsExtensions.
%   \caption{Simulation Results}
%   \label{fig_sim}
% \end{figure}

% Note that IEEE typically puts floats only at the top, even when this
% results in a large percentage of a column being occupied by floats.


% An example of a double column floating figure using two subfigures.
% (The subfig.sty package must be loaded for this to work.)
% The subfigure \label commands are set within each subfloat command, the
% \label for the overall figure must come after \caption.
% \hfil must be used as a separator to get equal spacing.
% The subfigure.sty package works much the same way, except \subfigure is
% used instead of \subfloat.
%
% \begin{figure*}[!t]
%   \centerline{\subfloat[Case I]\includegraphics[width=2.5in]{subfigcase1}%
%   \label{fig_first_case}}
%   \hfil
%   \subfloat[Case II]{\includegraphics[width=2.5in]{subfigcase2}%
%   \label{fig_second_case}}}
%   \caption{Simulation results}
%   \label{fig_sim}
% \end{figure*}
%
% Note that often IEEE papers with subfigures do not employ subfigure
% captions (using the optional argument to \subfloat), but instead will
% reference/describe all of them (a), (b), etc., within the main caption.


% An example of a floating table. Note that, for IEEE style tables, the
% \caption command should come BEFORE the table. Table text will default to
% \footnotesize as IEEE normally uses this smaller font for tables.
% The \label must come after \caption as always.
%
% \begin{table}[!t]
%%   increase table row spacing, adjust to taste
%   \renewcommand{\arraystretch}{1.3}
%   if using array.sty, it might be a good idea to tweak the value of
%   \extrarowheight as needed to properly center the text within the cells
%   \caption{An Example of a Table}
%   \label{table_example}
%   \centering
%%   Some packages, such as MDW tools, offer better commands for making tables
%%   than the plain LaTeX2e tabular which is used here.
%   \begin{tabular}{|c||c|}
%     \hline
%     One & Two\\
%     \hline
%     Three & Four\\
%     \hline
%   \end{tabular}
% \end{table}


% Note that IEEE does not put floats in the very first column - or typically
% anywhere on the first page for that matter. Also, in-text middle ("here")
% positioning is not used. Most IEEE journals/conferences use top floats
% exclusively. Note that, LaTeX2e, unlike IEEE journals/conferences, places








\section{Evaluation}

\subsection{Performance}

Can we create efficient designs, with high performance?  Yes, we
evaluate our approach by writing a design for RTM. This achieves
performance equal to state-of-the art published results.

\subsection{Ease of Use}

Is the language easy to use?
Yes, it is intuitive based on C constructs.

\subsection{Code Reuse, Maintainability etc.}



RTM

Scan ocean floor to find oil and gas [Surface? HUGE]
Huge data volume, complex physics [??]

Isotropic vs anisotropic

* Isotropic
* TTI
* VTI
* 30 Hz vs 60 Hz

* Convolution
* Sparse Matrix
