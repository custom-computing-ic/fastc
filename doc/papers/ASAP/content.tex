\section{Introduction}

 Custom designs on FPGA devices can be used to improve performance and
power consumption of computationally intensive algorithms. However,
creating custom designs is a difficult task and requires detailed
knowledge of circuit design and underlying performance
considerations. Ideally we want to synthesize custom designs from
existing high level implementations in C or other languages without
compromising on performance and energy efficiency.

We introduce a new C based meta-programming language and API for


We propose a compilation flow from high-level C programs to a
streaming dataflow architecture that captures and applies
optimizations using LARA, an aspect oriented programming language for
embedded systems [??]

Our contributions include:
\begin{enumerate}

\item We introduce MaxC, a C based meta-programming language and API
  for specifying dataflow designs;

\item We propose a DSE method based on MaxC and LARA for exploring
  optimizations of streaming data flow kernel designs;

\item We evaluate our approach using it to generate various
  implementations of a high performance application based on the
  Reverse Time Migration (RTM) technique for seismic imaging [??].

\end{enumerate}

\section{Related Work}

\subsection{Dataflow Languages}
\begin{itemize}

\item MaxJ
\item StreamIt
\item Streams-C


\end{itemize}

\subsection{Aspect Driven Desgin}

\subsection{Something about hardware?}

\section{Proposed design flow}

We perform a DSE step until we find a design that fits the chip,
achieves timing closure and meets non functional requirements.  For
each optimization strategy in our repository we:

\begin{enumerate}

\item apply the set of specific low-level optimizations comprised in
  the strategy (e.g. setting DSP balance).

\item  we compile the optimized
  design using the MaxCC backend to MaxJava, Maxeler's own dataflow
  language

\item we start the backend compilation toolchain (MaxCompiler
  and Xilinx) and perform an analysis of the reporting information,
  based on which we either restart the flow with the next optimization
  strategy step or proceed to the next step

\item we measure NFRs such as performance or latency and if our target
  is not met we restart using a different optimization strategy

\end{enumerate}

\begin{center}
\begin{tikzpicture}[node distance = 2cm, auto]
    \node [block] (csrc) {C source};
    \node [block, below of=csrc, left of=csrc] (maxrt) {CPU Runtime code};
    \node [block, below of=csrc, right of=csrc] (maxc) {MaxC Design};
    \node [block, below of=maxc] (maxj) {MaxJ Design};
    \node [block, below of=maxj] (maxfile) {Maxfile};
    \node [block, below of=maxfile, left of=maxfile] (app) {Application executable};
    \node [block, below of=app] (maxnode) {MaxNode, MaxStation};

    \node [cloud, right of=csrc] (lara) {LARA};
    \node [cloud, left of=csrc] (maxcc) {MaxCC};
%   \node [cloud, right of=init] (system) {system};
%    \node [block, below of=init] (identify) {identify candidate models};
%    \node [block, below of=identify] (evaluate) {evaluate candidate models};
%    \node [block, left of=evaluate, node distance=3cm] (update) {update model};
%    \node [decision, below of=evaluate] (decide) {is best candidate better?};
%    \node [belock, below of=decide, node distance=3cm] (stop) {stop};

    \path [line] (csrc) -- (maxrt);
    \path [line] (csrc) -- (maxc);
    \path [line] (maxc) -- (maxj);
    \path [line] (maxj) -- (maxfile);
    \path [line] (maxfile) -- (app);
    \path [line] (maxrt) -- (app);
    \path [line] (app) -- (maxnode);
    \path [line, dashed] (lara) -- (csrc);
    \path [line, dashed] (lara) -- (maxc);

%    \path [line] (decide) -| node [near start] {yes} (update);
%    \path [line] (update) |- (identify);
%    \path [line] (decide) -- node {no}(stop);
%    \path [line,dashed] (expert) -- (init);
%    \path [line,dashed] (system) -- (init);
%    \path [line,dashed] (system) |- (evaluate);
\end{tikzpicture}
\end{center}

% An example of a floating figure using the graphicx package.
% Note that \label must occur AFTER (or within) \caption.
% For figures, \caption should occur after the \includegraphics.
% Note that IEEEtran v1.7 and later has special internal code that
% is designed to preserve the operation of \label within \caption
% even when the captionsoff option is in effect. However, because
% of issues like this, it may be the safest practice to put all your
% \label just after \caption rather than within \caption{}.
%
% Reminder: the "draftcls" or "draftclsnofoot", not "draft", class
% option should be used if it is desired that the figures are to be
% displayed while in draft mode.
%
%\begin{figure}[!t]
%\centering
%\includegraphics[width=2.5in]{myfigure}
% where an .eps filename suffix will be assumed under latex,
% and a .pdf suffix will be assumed for pdflatex; or what has been declared
% via \DeclareGraphicsExtensions.
%\caption{Simulation Results}
%\label{fig_sim}
%\end{figure}

% Note that IEEE typically puts floats only at the top, even when this
% results in a large percentage of a column being occupied by floats.


% An example of a double column floating figure using two subfigures.
% (The subfig.sty package must be loaded for this to work.)
% The subfigure \label commands are set within each subfloat command, the
% \label for the overall figure must come after \caption.
% \hfil must be used as a separator to get equal spacing.
% The subfigure.sty package works much the same way, except \subfigure is
% used instead of \subfloat.
%
%\begin{figure*}[!t]
%\centerline{\subfloat[Case I]\includegraphics[width=2.5in]{subfigcase1}%
%\label{fig_first_case}}
%\hfil
%\subfloat[Case II]{\includegraphics[width=2.5in]{subfigcase2}%
%\label{fig_second_case}}}
%\caption{Simulation results}
%\label{fig_sim}
%\end{figure*}
%
% Note that often IEEE papers with subfigures do not employ subfigure
% captions (using the optional argument to \subfloat), but instead will
% reference/describe all of them (a), (b), etc., within the main caption.


% An example of a floating table. Note that, for IEEE style tables, the
% \caption command should come BEFORE the table. Table text will default to
% \footnotesize as IEEE normally uses this smaller font for tables.
% The \label must come after \caption as always.
%
%\begin{table}[!t]
%% increase table row spacing, adjust to taste
%\renewcommand{\arraystretch}{1.3}
% if using array.sty, it might be a good idea to tweak the value of
% \extrarowheight as needed to properly center the text within the cells
%\caption{An Example of a Table}
%\label{table_example}
%\centering
%% Some packages, such as MDW tools, offer better commands for making tables
%% than the plain LaTeX2e tabular which is used here.
%\begin{tabular}{|c||c|}
%\hline
%One & Two\\
%\hline
%Three & Four\\
%\hline
%\end{tabular}
%\end{table}


% Note that IEEE does not put floats in the very first column - or typically
% anywhere on the first page for that matter. Also, in-text middle ("here")
% positioning is not used. Most IEEE journals/conferences use top floats
% exclusively. Note that, LaTeX2e, unlike IEEE journals/conferences, places
% footnotes above bottom floats. This can be corrected via the \fnbelowfloat
% command of the stfloats package.


\section{The  MaxC Language}

MaxC is a programing language and API based on the C99 standard. It
provides a succinct means of specifying dataflow designs and is an
intermediate representation in the proposed design flow. MaxC is a
meta-programming language in the sense that, unlike a regular C
program, a MaxC program is not "executed"" but specifies
a hardware design. MaxC is designed to be human readable and to
facilitate integration with other tools. Being based on the C99
standard, the language interacts well with existing compilers or source
to source translation frameworks (e.g. LARA, ROSE), allowing source
level optimizations to be applied through different tools.

The example below presents a simple kernel performing a 1d
convolution.

Lines 2-6 show the kernel declaration specifying the stream input "p"
and output "out" as well as a number of scalar parameters configurable
at run-time.

On lines 13-16 we use array index notation to access future (positive
offset) or past (negative offset) stream elements.

Normal arithmetic operations can be used to operate on stream
elements.

\lstset{style=MaxC}

\begin{lstlisting}
#pragma stream:result type:uint32 func:kernel_Convolution1d
void kernel_Convolution1d(
  sin_float8_24 p,
  sout_float8_24 out,
  float8_24 c_0_0_0, float8_24 c_p_0_0, float8_24 c_n_0_0,
  uint32 n1, uint32 ORDER)
{
    s_float8_24 i4 = count(1000, 1);
    s_float8_24 i1 = countChain(n1, 1, i4);

    s_int result =
        p[0]  * c_0_0_0 +
        p[1]  * c_p_0_0 +
        p[-1] * c_n_0_0;

    s_bool up = (i1 >= ORDER) && (i1 < n1 - ORDER);
    s_int32 inter = stream_select(up, p, result);

    output_i(out, inter);
}
\end{lstlisting}

The example illustrates the additional types we introduce as
extensions to C99. These are meant to simplify the language and make
it more expressive as well as provide additional verification and
optimisation mechanisms (e.g. type safety). Maintaining compatibility
with C99 simplifies the kernel simulation flow as discussed below.

\section{Implementation}

The MaxCC backend is implemented using the ROSE compiler framework
[@rose00]. It processes the Abstract Syntax Tree generated by the ROSE
frontend into a data flow graph representation of the original MaxC
design. It then traverses this graph in conjunction with the original
AST emitting MaxJava code.

MaxCC supports the following types, as extensions to C99:

\begin{itemize}
\item floati\_j (e.g float8\_24, float10\_22 etc.), uinti/inti
  (e.g. uint32, int32) allow specification of custom width data
  types

\item s\_floati\_j, s\_inti\_j etc. (e.g. s\_float8\_24) allow specification of
  stream data types
\end{itemize}

To allow decoupling of bit width specifications from the functional
code we also provide the type s\_float, s\_int etc. In this
situation we can specify the type using a pragma directive.

The previous example also illustrates some of the MaxC API components:

 * output functions (e.g. output\_i) are used to connect an internal
   kernel stream to the output stream of a kernel

 * various counters and counter chain configurations can be
   instantiated using functions from the counter API as shown on lines
   8-9

 * functions such as stream\_select are used to multiplex between a
   number of streams based on the value of condition stream

\section{Kernel Simulation}

The goal for the kernel simulation model is that it should be possible
to compile and run it using the standard GCC toolchain, verifying the
logical correctness of the design (i.e. not accounting for hardware
effects such as stalls etc.).

To achieve this we provide the type extensions described above as type
definitions. Streams are modelled as pointers: `typedef float*
s\_float8\_24`. After each kernel cycle, stream pointers are
incremented, as shown in the simulation loop below:

\lstset{style=MaxC}

\begin{lstlisting}
for (int i = 0 ; i < CYCLE_COUNT; i++) {
   kernel_Convolution1d(p, out, 2);
   update_stream_pointers(p, out);
}
\end{lstlisting}

The major limitation of this approach is that, due to the lack of
support for polymorphism in C, in order to enable simulation support,
users are required to add boilerplate code, that would otherwise not
be required by the MaxCC backend. When compiling with MaxCC this
boilerplate code is simply ignored, so any mistakes in this code will
not be captured by the simulation model.

This issue is illustrated in the example below where we must use the
`stream\_init\_i(int stream\_id)` function, passing in a unique stream id
to retrieve the appropriate stream values from the global data
structures they are stored in. The function then either allocates a
new stream or increments the stream pointer if the stream has already
been allocated. Furthermore if we extend the example to use two
counters, we similarly require unique identifiers for counters.

\begin{lstlisting}
void kernel_Convolution1d(
    sin_float8_24 in,
    sout_float8_24 out,
    float8_24 c)
{
    float8_24 count = counter(10, 1);

    s_int32 result = stream_init_i(0);
	result[0] = c * (in[0] + in[1]);

    int32 good = (count > 2) && (count < 5);

    int32 inter = stream_selectfi(good, in, result);

    output_i(out, inter);
}

\end{lstlisting}

\section{Connecting Kernels}

We create a simple configuration language for specifying kernel and
host interconnections. The configuration file used for the RTM
application is shown as an example below. This allows us to specify
the kernel instances that form our design. The configuration file has
two parts. First we specify the kernel instances. For example line 3
in the listing below specifies that we will be creating 3 kernel
instances based on a kernel named Cmdread. After specifying all kernel
instances, we can specify the flow of data between kernels, host and
memory. We can specify three types of flow:

* kernel to kernel for example `Myapp0[out] > Myapp1[in]`. This will
  connect the output of Myapp0 named "out" to the input of Myapp1
  named "in";

* memory to kernel, as shown on line 10 in the listing below; this
  will connect a dram stream named "dram2knl0" to the input of Myapp0
  "in"; the read sequence is provided by the output of Cmdread0 named
  dram\_read; we can also specify a kernel to memory direction, as
  shown on line 15 in which case the second argument specifies the
  stream providing the write sequence;

* finally, we can create host to memory and memory to host connections
  as shown on lines 7 - 8.

\lstset{style=MaxCconf}

\begin{lstlisting}
kernels:
  Cmdread:  Cmdread0 Cmdread1 Cmdread2 Cmdread3 Cmdhostread
  Cmdwrite: Cmdwrite0 Cmdwrite1 Cmdhostwrite
  RTM:      MyApp0

flow:
  Host > Memory[dram2mgr, Cmdhostwrite.dram_write]
  Memory[mgr2dram, Cmdhostread.dram_read] > Host

  Memory[dram2knl0, Cmdread0.dram_read] > Myapp0[burst_p]
  Memory[dram2knl1, Cmdread1.dram_read] > Myapp0[burst_pp]
  Memory[dram2knl2, Cmdread2.dram_read] > Myapp0[burst_dvv]
  Memory[dram2knl3, Cmdread3.dram_read] > Myapp0[burst_source]

  MyApp0[ker_p] > Memory[knl2dram0, Cmdwrite0.dram_write]
  MyApp0[output_pp] > Memory[knl2dram1, Cmdwrite1.dram_write]
\end{lstlisting}


\section{Evaluation}

\subsection{Performance}

Can we create efficient designs, with high performance?  Yes, we
evaluate our approach by writing a design for RTM. This achieves
performance equal to state-of-the art published results.

\subsection{Ease of Use}

Is the language easy to use?
Yes, it is intuitive based on C constructs.

\subsection{Code Reuse, Maintainability etc.}



RTM

Scan ocean floor to find oil and gas [Surface? HUGE]
Huge data volume, complex physics [??]

Isotropic vs anisotropic

* Isotropic
* TTI
* VTI
* 30 Hz vs 60 Hz

* Convolution
* Sparse Matrix
