\section{Aspects}
\label{sec:aspects}

Aspects are standalone modules that capture functional cross-cutting concerns
that are decoupled from the primary function a program. AspectJ~\cite{Kiczales:2001}, which is the standard language for Aspect-Oriented Programming (AOP),  captures program execution points (such as method calls) at run-time to allow new code to be executed before, after or in place of these execution points through a process called \emph{weaving}. The main motivation behind AspectJ in particular, and AOP in general, is to solve the modularization problem when dealing with multiple cross-cutting functional concerns.

The LARA aspect-oriented design-flow~\cite{Cardoso:Carvalho:Cutinho:Luk:Nobre:Diniz:Petrov:2012}, on the other hand, performs the weaving process at compile-time to satisfy non-functional concerns, such as to improve performance on a particular hardware platform. For this purpose, the LARA weaving process manipulates and transforms the application sources. These new generated sources (woven code) incorporate functional elements of the original sources, and non-functional concerns captured by LARA aspects.

In this paper, we combine the LARA aspect design-flow with MaxC dataflow designs. In particular, MaxC uses standard C99 syntax to capture dataflow computations while aspects specify decoupled optimization and transformation strategies that operate on MaxC descriptions. This makes the
functionality of the application easier to understand, more
maintainable and portable since it is no longer obscured by various
structural or algorithmic transformations, and platform specific
optimizations. In addition, strategies coded in LARA can be re-applied automatically in different applications, thus improving design productivity.

In this paper, we report four types of aspects (Table~\ref{tbl:aspects}) used with MaxC designs:

\vspace*{0.5ex}
\noindent{\bf System Aspects.} System aspects capture transformation or optimization strategies that affect the whole application such as those concerning
hardware/software partitioning, monitorisation and run-time reconfiguration
capabilities. The goal of hardware/software partitioning is to improve the overall
execution time by identifying parts of the code to be offloaded to hardware (Section~\ref{sect:asp_hsp}). Monitorisation aspects instrument the application code to extract run-time behaviour, and uncover opportunities for optimisation (Section~\ref{sect:asp_mon}). Runtime reconfiguration (Section~\ref{sect:asp_reconfig}) removes idle functions from the accelerator at specific points in time, to use more resources for functions that are active.

\vspace*{0.5ex}
\noindent{\bf Implementation Aspects.} Implementation aspects focus on low level design optimizations that can be applied to the MaxC designs in order to improve timing or resource usage. For instance, operator optimisation aspects (Section~\ref{sect:asp_ops}) can be used to map operators in the program to dedicated hardware resources. 

\vspace*{0.5ex}
\noindent{\bf Exploration Aspects.} Exploration aspects deal with strategies that generate multiple designs to find optimal design based on user requirements. Exploration aspects can act on any level of the design flow (C code, C and MaxC, or MaxC functions). They enable
systematic exploration of trade-offs and optimisation opportunities. Examples of exploration aspects include simple iterative and metaheuristic.

\vspace*{0.5ex}
\noindent{\bf Development Aspects.} Development aspects capture transformations that have an impact on the
development process such as debugging, simulating kernels or improving
compilation speed. Separating these concerns makes the original
application code easier to maintain and enables the automatic
application of these transformations to a wide range of designs, thus
improving developer productivity.

\begin{table}[tp]
\caption{Types of Aspects used in MaxC}
\label{tbl:aspects}
\centering
\begin{tabular}{l|l|l}
\hline
\bf{Aspect Type} & \bf{Aspect Name} & \bf{Description} \\
\hline
\hline
\multirow{3}{*}{system} & \blt hw/sw partitioning & capture mapping between  \\
                        & \blt monitorisation & application modules and \\
                        & \blt reconfiguration & GPP/GPU/FPGA accelerators\\
\hline
\multirow{2}{*}{implementation} &\blt operator optimisation &  capture low-level hardware \\
& \blt design configuration & optimisations  \\
\hline
\multirow{3}{*}{exploration} & \multirow{2}{*}{\blt metaheuristic} & generate multiple implemen- \\
 & \multirow{2}{*}{\blt iterative} & tations based on design  \\
 & & space exploration strategies \\
\hline
\multirow{3}{*}{development} & \blt simulation & \multirow{2}{*}{improve developer}  \\
& \blt debugging & \multirow{2}{*}{productivity} \\
& \blt compilation &  \\
\hline
\end{tabular}
\end{table}

\subsection{HW/SW partitioning}
\label{sect:asp_hsp}
MaxC functions describing dataflow computations can be embedded within the C application but cannot be invoked directly by software C functions. 
Instead, a MaxC pragma must be used on top of software functions definitions or C calls to indicate an alternate hardware implementation. For instance, the following C code:

\noindent\texttt{\footnotesize\marktext{void maxc\_f() \{\emph{/* dataflow implementation */}\}} \\
void f() \{\emph{/* software implementation */}\} \\
\marktext{\#pragma maxc hw:maxc\_f} \\
f(); \\
}

\noindent indicates that the software implementation of f() can be mapped to the dataflow implementation described in maxc\_f(). This way, the maxcc compiler can automatically switch from 
a pure software application to a software/hardware design.

Hence, a hardware/software partitioning strategy can be performed in five steps: (i)~detecting hotspots in the program, (ii)~detecting code patterns from hotspots that are suited for dataflow computation and acceleration, (iii)~performing the outlining transformation so that each candidate for acceleration is enclosed in a function $f$, (iv)~deriving a dataflow version $max\_f$ from state-based $f$, (v)~placing a maxc pragma on top of each function call to $f$ and associate it to the corresponding $maxc\_f$ function. Each of these steps can be described as separate LARA aspect and combined to form a hardware/software partitioning strategy. 

\subsection{Monitorisation Aspect}
\label{sect:asp_mon}
To find potential hotspots in the application, for instance to perform hardware/software partitioning, we can use the aspect in Fig.~\ref{fig:hotspot}:

\lstset{style=lara}
\begin{figure}[!h]
\begin{lstlisting}
aspectdef LoopMonitor
select function.loop{is_innermost} end
apply
    $loop.insert before
       %{monitor_instanceI("[[$loop.key]]");}%;
    $loop.insert after
       %{monitor_instanceE("[[$loop.key]]");}%;
end

select function.loop{is_innermost}.entry end
apply $begin.insert after
       %{monitor_iterI("[[$loop.key]]");}%;
end
select function.loop{is_innermost}.exit end
apply $begin.insert before
       %{monitor_iterE("[[$loop.key]]");}%;
end
end
\end{lstlisting}
\caption{Aspect that instruments the application to monitor loop activity. The information generated can be used to identify hotspots.}
\label{fig:hotspot}
\end{figure}

\noindent With the aspect in Fig.~\ref{fig:hotspot}, the weaver can automatically instrument any C application to self-monitor its innermost loops at run-time, as they are natural candidates for dataflow-based acceleration. In particular, this monitorization aspect can compute the following information for every innermost loop: (a) the average number of times it has been executed, (b) the average number of iterations, (c) the loop average time, and (d) the loop iteration average time. For this purpose, we use a monitoring API composed by 4 functions to mark the beginning and end of the loop (monitor\_instanceI and monitor\_instanceE respectively), and to mark the beginning and end of an iteration (monitor\_iterI and monitor\_iterE respectively). These monitoring functions keep an account of the frequency of execution and the time to complete the whole loop and a single iteration.  The aspect code is as follows: line~2 selects all loops in the application that are innermost (loops with no other loops enclosed); lines~3--8 place an instance monitor call before and after each selected loop; lines 10~13 select all entry points inside the loop and insert a monitoring call to mark the beginning of each iteration; lines~14--17 place an instance monitor call to mark the end of each iteration. The following table shows an example of applying the aspect from Fig.~\ref{fig:hotspot} on a C code containing a loop:

{\footnotesize
\fontfamily{pcr}\selectfont
\begin{tabular}{l|l}
\hline
\bf{original code} & \bf{woven code}  \\
\hline
\hline
void f() \{ & void f() \{ \\
%\hspace{3ex}$\ldots$         & \hspace{3ex}$\ldots$ \\
                             & \hspace{3ex}\marktext{monitor\_instanceI("f:1");} \\
\hspace{3ex}while (i < N) \{ & \hspace{3ex}while (i < N) \{ \\
                             & \hspace{6ex}\marktext{monitor\_iterI("f:1");} \\
\hspace{6ex}i++;             & \hspace{6ex}i++; \\
                             & \hspace{6ex}\marktext{monitor\_iterE("f:1");} \\
\hspace{3ex}\}               & \hspace{3ex}\} \\
                             & \hspace{3ex}\marktext{monitor\_instanceE("f:1");} \\
%\hspace{3ex}$\ldots$         & \hspace{3ex}$\ldots$ \\
\hline
\end{tabular}
}
\vspace{2ex}

\noindent It can be seen in the woven code that each monitoring call receives as a parameter the loop key, which uniquely identifies the loop within the application. The loop key concatenates the function name with the position of the loop within the abstract syntax tree. For instance, \emph{f:2:1} corresponds to the 1st loop inside the 2nd outermost loop of function $f$. The hotspots can be identified by an aspect (not shown) that takes the profiling information generated by the monitorization API calls, and uses an heuristic to compute the most profitable computations to be offloaded to hardware.

\subsection{Reconfiguration Aspect}
\label{sect:asp_reconfig}
To support runtime reconfiguration, we specify the configuration associated with the function call in the MaxC pragma. For instance:

\noindent\texttt{\footnotesize{\marktext{\#pragma maxc hw:maxc\_f0 cfg:c0}\\
x = f(0); \\
\marktext{\#pragma maxc hw:maxc\_f1 cfg:c1}\\
y = f(x); \\
\marktext{\#pragma maxc hw:maxc\_g cfg:c1}\\
z = g(x); \\
}}

\noindent With the above code annotations, the maxcc compiler can generate multiple configurations, each containing  a number of MaxC function implementations that can be executed in parallel. If the configuration name is not specified using the MaxC pragma, then the maxcc compiler assumes a default configuration. Having a single configuration can lead to situations where at any point in time and due to data dependencies, part of the functions are idle. With run-time reconfiguration, we can exploit unused resources to support active functions. In particular, during the execution of an application, we select various configurations at different points in time to maximize the utilization of FPGA resources. Within this context, we use a hardware partition, which contains a set of configurations that are used to support reconfiguration during the life cycle of an application. In the above example, configuration c0 contains a single implementation of f (\emph{maxc\_f0}), and thus can potentially use more resources and be faster than the \emph{maxc\_f1} version which must share the same configuration (\emph{c1}) with \emph{maxc\_g}.

The work in~\cite{Xinyu:Qiwei:Luk:Qiang:Pell:2012} proposes an
approach for extracting valid and efficient hardware partitions. To realize
runtime reconfiguration with maxcc without modifiying the original code we use the
aspect shown in Fig.~\ref{fig:aspect-reconf}.  The input to the
aspect is a hardware partition (lines 2--4). The partition is implemented as a
hash table that maps a function call (key) to a hardware
implementation, represented as tuple containing the hardware
implementation name (hw) and the associated configuration (cfg).

\lstset{style=lara}
\begin{figure}[!h]
\begin{lstlisting}
aspectdef AspReconfig
input
   partition
end
select function.call end
apply
   if ($call.key in partition) {
      var cfg = partition[$call.key].cfg;
      var hw = partition[$call.key].hw;
      $call.insert before %{
         #pragma maxc hw:[[hw]] cfg:[[cfg]]
      }%;
   }
end
end
\end{lstlisting}
\caption{Reconfiguration aspect.}
\label{fig:aspect-reconf}
\end{figure}

Table \ref{fig:aspect-hash} shows an example of a hash table representing 
a hardware partition. The key (e.g. main:f:1) identifies a function call in the application, and is
formed by concatenating the name of the caller function (main), the name of the invoked function (e.g. f) and a
unique number (1).  Line~5 in the aspect shown in Fig.~\ref{fig:aspect-reconf} selects all function calls,
and for each call found in the partition (line~7), we set the
appropriate pragma on top of the call statement (lines 10--12). We can now realize and experiment 
various reconfiguration designs by invoking this aspect with different hardware partitions.


%{\footnotesize
%\fontfamily{pcr}\selectfont
\begin{table}[!h]
\caption{An example of a hardware partition, represented as a hash table, used in the reconfiguration aspect (Fig.~\ref{fig:aspect-reconf})}
\label{fig:aspect-hash}
\centering
\begin{tabular}{c|c|c}
\hline
\multicolumn{3}{c}{\bf{partition}} \\
\hline
\bf{\$call.key} & \bf{hw} & \bf{cfg}  \\
\hline
main:f:1 & maxc\_f0 & c0 \\
main:f:2 & maxc\_f1 & c1 \\
main:g:3 & maxc\_g & c1 \\
\hline
\end{tabular}
\end{table}
%}

\subsection{Operator Optimisation Aspect}
\label{sect:asp_ops}
Aspects can be used to enable design space exploration of resource
usage vs. accuracy trade-offs by varying the word
lengths. Alternatively we can use them to explore resource trade-offs
by mapping computation to specialised FPGA blocks. For example, the
aspect in Fig.~\ref{fig:aspect-DSP} can be used to map arithmetic
expressions to DSP blocks. We define the DSP balance factor to be used
(none, normal, full), the granularity (which decides if we break up
arithmetic expressions to insert DSP Factors between them) and the
number of independent applications.

\lstset{style=lara}
\begin{figure}[!h]
  \centering
  \begin{lstlisting}
aspectdef dspFactor
var op_granularity =
 [{DspFactor: 'full', MultiplyOp: 20, AddOp: 5 },
  {DspFactor: 'normal', MultiplyOp: 3}];

select function.statement end
apply
   for (var i in op_granularity) {
      var gprofile = op_granularity[i];
      var match = true;
      for (var k in gprofile) {
         if (k != 'DspFactor') {
            match &= ($statement.num_construct(k)
                      >= gprofile[k]);}}
      if (match) {
         var pragma = '#pragma maxc balanceDSP:'
                      + gprofile.DspFactor;
         $statement.insert before "[[pragma]]";
         break;}}
   end
end
  \end{lstlisting}
  \caption{Aspect for exploring mapping of computation to DSP blocks.}
  \label{fig:aspect-DSP}
\end{figure}

Aspects steps:

\begin{enumerate}
\item Identify arithmetic expressions;
\item Split them according to granularity;
\item Insert annotation to adjust DSP usage;
\item Repeat for a given number of applications.
\end{enumerate}

\begin{figure}
  \begin{lstlisting}
    #pragma balanceDSP=full
    int result = p[0]  * c_0_0_0;
    int result = result + p[1]  * c_p_0_0;
    #pragma balanceDSP=full
    int result = result + p[-1] * c_n_0_0;
  \end{lstlisting}
  \caption{One possible result of applying
    \texttt{OptimizeDSPUsage(full, fine, 2)} to Lines 13--17 of Fig.~\ref{fig:maxc-1dconv}.}
  \label{fig:maxc-1dconv-aspect}
\end{figure}

Fig.~\ref{fig:maxc-1dconv-aspect} shows one of the possible results
of applying the aspect \texttt{OptimizeDSPUsage(full, fine, 2 } to
Lines 13--17 of Fig.~\ref{fig:maxc-1dconv}.

Using LARA we can implement and combine these aspects to enable
systematic design space exploration of all the optimisations options
exposed by the MaxC backend resulting in the generation of a large
number of designs. The feedback-directed compilation process of LARA can
be used to capture and extract feedback from the backend reports
pertaining to resource usage or timing information and automatically
adjust the compilation process.

\subsection{Iterative Design Exploration Aspect}

 An example LARA aspect for design space exploration is
shown in Fig.~\ref{fig:aspect-exploration}. It highlights the feedback capabilities of the design
flow: the aspect will generate and build the MaxC designs until the
resource usage passes a specified LUT threshold, at each step
increasing the parallelism of the design (by replicating the
computational pipeline).
% For instance, we can generate

\lstset{style=lara}
\begin{figure}[!h]
\begin{lstlisting}
  aspectdef DSEStrategy
  var config = { 'Dsp_factor': 1,
    'Exponent'  : 8,
    'Mantissa'  : 24 };
  var par = 0, lut_threshold = 10000;
  do {
    par++;
    config['Par'] = par;
    var designName = genName(config);
    call genMaxC(designName, config);
    buildMaxC(designName);
  } while (@hw[designName].lut < lut_threshold);
  end
\end{lstlisting}
\caption{Aspect for increasing design parallelism subject to a LUT
  threshold.}
\label{fig:aspect-exploration}
\end{figure}


\subsection{Debugging Aspect}

Because the current execution model does not provide runtime debugging
of hardware designs, the easiest solution to debug designs is to log
the values of various streams during execution. The insertion of debug
statements can be encapsulated in aspects. It is particularly
important to separate debug aspects from the original application code
since debug blocks can influence the compilation time and timing
constraints as well as the behaviour of the design (Fig~\ref{fig:aspect-debug}).

\lstset{style=lara}
\begin{figure}[!h]
  \centering
\begin{lstlisting}
aspectdef DebugValues

select function.vref end
apply
   $vref.parent_stmt.insert before
    %{ log("[[$vref.name]]", [[$vref.name]]); }%
   $vref.parent_stmt.insert after
    %{ log("[[$vref.name]]", [[$vref.name]]); }%
end
condition $vref.is_out end

end
\end{lstlisting}
  \caption{Aspect for automatically inserting debug blocks.}
  \label{fig:aspect-debug}
\end{figure}


