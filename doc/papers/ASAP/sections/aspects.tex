\section{Aspects}

Aspects are standalone modules used to capture cross-cutting concerns
(such as logging or platform specific optimizations) and separate them
from the original application code \cite{Kiczales:2001}. An aspect
weaver reads the modules and generates appropriate code which is
injected at the specified points in the source application. MaxC was
designed to integrate with aspect weaving tools by using standard C99
syntax. This allows MaxC specifications to describe the simplest form
of a dataflow design while aspects specify decoupled optimization and
transformation strategies that operate on this design. This make the
functionality of the application easier to understand, more
maintainable and portable since it is no longer obscured by various
structural or algorithmic transformations or platform specific
optimizations.

We distinguish four categories of aspects: \emph{system aspects},
\emph{exploration aspects}, \emph{implementation aspects} and
\emph{development aspects}.


\subsection{System Aspects}

System aspects capture transformation or optimization strategies that
affect the whole application such as those concerning
hardware/software portioning, or run-time reconfiguration
capabilities. The goal is to expose parallelism, improve communication
between CPU and accelerator via hardware/software partitioning or
remove idle functions from the accelerator using run-time
reconfiguration. These aspects act on the original C application level
as well as on the partitioned C (software) and MaxC design (hardware).


\subsubsection{\TODO Reconfiguration}

\subsection{Exploration aspects}

Exploration deal with strategies that generate multiple designs based
on parametrisation and can act on any level of the design flow (C
code, C and MaxC, or MaxC design). They enable systematic exploration
of trade-offs and. An example LARA aspect for design space exploration
is show below. It highlights the feedback capabilities of the design
flow: the aspect will generate a MaxC designs and build the DFE
configuration until this passes a specified LUT threshold, at each
step increasing the parallelism of the design (by replicating the
computational pipeline).

\lstset{style=lara}
\begin{lstlisting}
  aspectdef DSEStrategy
  var config = { 'Dsp_factor': 1,
    'Exponent'  : 8,
    'Mantissa'  : 24 };
  var par = 0, lut_threshold = 10000;
  do {
    par++;
    config['Par'] = par;
    var designName = genName(config);
    call genMaxC(designName, config);
    buildMaxC(designName);
  } while (@hw[designName].lut < lut_threshold);
  end
\end{lstlisting}

\subsection{Implementation Aspects}

Design aspects capture low level design optimizations that can be
applied to the MaxC designs in order to improve timing or explore
various resource usage trade-offs.

Aspects can be used to enable design space exploration of resource
usage vs. accuracy trade-offs by varying the word
lengths. Alternatively we can use them to explore resource trade-offs
by mapping computation to specialised FPGA blocks. For example, the
aspect in Figure \ref{fig:aspect-DSP} can be used to map arithmetic
expressions to DSP blocks. We define the DSP balance factor to be used
(none, normal, full), the granularity (which decides if we break up
arithmetic expressions to insert DSP Factors between them) and the
number of independent applications.

\lstset{style=lara}
\begin{figure}[!h]
  \centering
  \begin{lstlisting}
aspectdef dspFactor
var op_granularity =
 [{DspFactor: 'full', MultiplyOp: 20, AddOp: 5 },
  {DspFactor: 'normal', MultiplyOp: 3}];

select function.statement end
apply
   for (var i in op_granularity) {
      var gprofile = op_granularity[i];
      var match = true;
      for (var k in gprofile) {
         if (k != 'DspFactor') {
            match &= ($statement.num_construct(k)
                      >= gprofile[k]);}}
      if (match) {
         var pragma = '#pragma maxc balanceDSP:'
                      + gprofile.DspFactor;
         $statement.insert before "[[pragma]]";
         break;}}
   end
end
  \end{lstlisting}
  \caption{Aspect for exploring mapping of computation to DSP blocks.}
  \label{fig:aspect-DSP}
\end{figure}

Aspects steps:

\begin{enumerate}
\item Identify arithmetic expressions;
\item Split them according to granularity;
\item Insert annotation to adjust DSP usage;
\item Repeat for a given number of applications.
\end{enumerate}

\newsavebox{\secondlisting}
\begin{lrbox}{\secondlisting}% Store second listing
  \lstinputlisting[firstline=8]{code.txt}
\end{lrbox}

\begin{figure}
  \begin{lstlisting}
    #pragma balanceDSP=full
    int result = p[0]  * c_0_0_0;
    int result = result + p[1]  * c_p_0_0;
    #pragma balanceDSP=full
    int result = result + p[-1] * c_n_0_0;
  \end{lstlisting}
  \caption{One possible result of applying
    \texttt{OptimizeDSPUsage(full, fine, 2)} to Lines 13 -- 17 of Figure
    \ref{fig:maxc-1dconv}.}
  \label{fig:maxc-1dconv-aspect}
\end{figure}

Figure \ref{fig:maxc-1dconv-aspect} shows one of the possible results
of applying the aspect \texttt{OptimizeDSPUsage(full, fine, 2 } to
Lines 13 -- 17 of Figure \ref{fig:maxc-1dconv}.

Using LARA we can implement and combine these aspects to enable
systematic design space exploration of all the optimisations options
exposed by the MaxC backend resulting in the generation of a large
number of designs. The feedback-directed compilation process of LARA can
be used to capture and extract feedback from the backend reports
pertaining to resource usage or timing information and automatically
adjust the compilation process.

\subsection{Development Aspects}

Development aspects capture transformations that have an impact on the
development process such as debugging, simulating kernels or improving
compilation speed. Separating these concerns makes the original
application code easier to maintain and enables the automatic
application of these transformations to a wide range of designs, thus
improving developer productivity.

\subsubsection{Simulation Aspects}

The goal for the MaxC kernel simulation model is that it should be
possible to compile and run dataflow designs using the standard GCC
toolchain in order to verify the logical correctness of the
design. However, this often leads to the need of adding boilerplate
code that would otherwise not be required by the MaxC backend, just to
enable simulation. Since this process is itself manual and hence
error-prone it defeats the very purpose of testing the design. One
approach would be to require users to always use the simulation
API. However this unnecessarily complicates the dataflow design. Our
solution is to use separate aspects to generate the simulation
designs.

Simulation aspects can be applied to the original dataflow design to
enable pure software simulation. This approach gives more freedom when
designing MaxC constructs -- since we are not necessarily concerned
weather they compile using the standard GCC toolchain. This in turn
allows us to provide a neater syntax but also hides and automates the
details of generating simulation designs from the developer.

Aspect steps:

\begin{enumerate}

\item insert kernel simulation loop in hostcode %(Figure )
  \begin{figure}[!h]
    \begin{lstlisting}
      for (int i = 0 ; i < CYCLE_COUNT; i++) {
        kernel_Convolution1d(p, out, 2);
        update_stream_pointers(p, out);
      }
    \end{lstlisting}
    %\caption{Stream simulation loop.}
    %\label{fig:maxc-simulation}
  \end{figure}

\item insert boilerplate code in the dataflow design to enable simulation.
  % The result of applying the simulation aspect to the dataflow in
  %  Figure \ref{fig:maxc-1dconv} is shown in Figure
  %  \ref{fig:maxc-sim-aspect}.
  For example any stream and counter used in the kernel body need to
  be declared using special calls that also pass in unique identifiers
  which are used to store and retrieve the values at each cycle in a
  separate data structure (e.g. \texttt{`stream\_init\_i(int
    stream\_id)}). On each kernel cycle, the function then either
  allocates a new stream or increments the stream pointer if the
  stream has already been allocated.
  \begin{comment}
    \begin{figure}[!h]
      \begin{lstlisting}
        float* i4 = count_i(1000, 1, 0);
        float* i1 = countChain_i(n1, 1, i4, 0);
      \end{lstlisting}
      \caption{Applying the simulation aspect to the dataflow design in
        Figure \ref{fig:maxc-1dconv}}
      \label{fig:maxc-sim-aspect}
    \end{figure}
  \end{comment}

\end{enumerate}

\lstset{style=MaxC}


\subsubsection{Debug Aspects}

Because the current execution model does not provide for runtime
debugging of hardware designs the easiest solution to debug designs is
to log the values of various streams during execution. The insertion
of debug statements can be encapsulated in aspects. It is particularly
important to separate debug aspects from the original application code
since debug blocks can influence the compilation time and timing
constraints as well as the behaviour of the design.

\lstset{style=lara}
\begin{figure}[!h]
  \centering
\begin{lstlisting}
aspectdef DebugValues
    
select function.vref end
apply
   $vref.parent_stmt.insert before 
    %{ log("[[$vref.name]]", [[$vref.name]]); }%
   $vref.parent_stmt.insert after 
    %{ log("[[$vref.name]]", [[$vref.name]]); }%
end
condition $vref.is_out end

end
\end{lstlisting}
  \caption{Aspect for exploring mapping of computation to DSP blocks.}
  \label{fig:aspect-DSP}
\end{figure}

\subsubsection{Compilation Aspects}

Compilation aspects can be applied during the development process to
create versions of the dataflow design that compile faster. They apply
changes such as reducing the operating frequency, removing debug
blocks or applying design-level optimizations that can resolve timing
constraints. Naturally reducing the compilation time increases
developer productivity.

Different compilation aspects can be used to build in parallel more
versions of the application that allow incremental testing (in order
of compilation times):
\begin{enumerate}
\item Fastest design to build will be a naive version, optimized for
  compile time (minimum design and memory frequency, disabled debug
  blocks, no arithmetic optimizations). This allows verifying
  application correctness;
\item The debug version (enabled debug blocks) is built in parallel to
  allow debugging of errors identified in the previous step;
\item Finally a version optimized for performance (maximum frequency,
  no debug block, arithmetic optimisations) is compiled to allow
  testing of non-functional requirement.
\end{enumerate}
