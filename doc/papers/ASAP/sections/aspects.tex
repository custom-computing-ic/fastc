\section{Aspects}
\label{sec:aspects}

Aspects are standalone modules that capture functional cross-cutting
concerns that are decoupled from the primary function of a
program. AspectJ\cite{Kiczales:2005:AP:1062455.1062640}, a widely
used extension for Aspect-Oriented Programming (AOP) in Java, captures
program execution points (such as method calls) at run-time to allow
new code to be executed before, after or in place of these execution
points through a process called \emph{weaving}. The main motivation
behind AspectJ in particular, and AOP in general, is to solve the
modularisation problem when dealing with multiple cross-cutting
functional concerns.

The LARA aspect-oriented
design-flow \cite{Cardoso:Carvalho:Cutinho:Luk:Nobre:Diniz:Petrov:2012},
on the other hand, performs the weaving process at compile-time to
satisfy non-functional concerns, such as to improve performance on a
particular hardware platform. For this purpose, the LARA weaving
process manipulates and transforms the application sources. These new
generated sources (woven code) incorporate functional elements of the
original sources, and non-functional concerns captured by LARA
aspects.

In this paper, we combine the LARA aspect design-flow with \MAXC{}
dataflow designs. In particular, \MAXC{} uses standard C99 syntax to
capture dataflow computations while aspects specify decoupled
optimisation and transformation strategies that operate on \MAXC{}
descriptions. This approach makes the functionality of the application
easier to understand, more maintainable and portable since it is no
longer obscured by various structural or algorithmic transformations,
as well as platform specific optimisations. In addition, strategies
coded in LARA can be re-applied automatically in different
applications, thus improving developer productivity.

We report four types of aspects (Table~\ref{tbl:aspects}) used with
designs in \MAXC{}:

\vspace*{1.2ex}
\noindent{\bf System Aspects.} System aspects capture transformation
or optimisation strategies that affect the whole application such as
those concerning hardware/software partitioning, monitorisation and
run-time reconfiguration capabilities. The goal of hardware/ software
partitioning is to improve the overall execution time by identifying
parts of the code to be offloaded to hardware
(Section~\ref{sect:asp_hsp}). Monitorisation aspects instrument the
application code to extract run-time behaviour, and uncover
opportunities for optimisation (Section~\ref{sect:asp_mon}).
Run-time reconfiguration can be used to remove idle functions
from the accelerator at specific points in time, so that
additional resources can be dedicated to functions that are
active \cite{Xinyu:Qiwei:Luk:Qiang:Pell:2012}.

\vspace*{1.2ex}
\noindent{\bf Implementation Aspects.} Implementation aspects focus on
low level design optimisations that can be applied to designs in
\MAXC{} to improve timing or resource usage. For instance, operator
optimisation aspects (Section~\ref{sect:asp_ops}) can be used to map
operators in the program to dedicated hardware resources. Word-length
aspects specify the numerical representation of variables and
expressions in the design.

\vspace*{1.2ex}
\noindent{\bf Exploration Aspects.} Exploration aspects deal with
strategies that generate multiple designs to find an optimal
implementation based on user requirements. Exploration aspects can act
on any level of the design flow (C code, C and \MAXC{}, or \MAXC{}
functions). They enable systematic exploration of trade-offs and
optimisation opportunities. Examples of exploration aspects include
iterative aspects (Section~\ref{sect:asp_it}) which generate a
sequence of solutions until a termination criterion is satisfied, and
metaheuristic-based aspects to find optimal solutions in a very large
design space.

\vspace*{1.2ex}
\noindent{\bf Development Aspects.} Development aspects capture
transformations that have an impact on the development process such as
debugging (Section~\ref{sect:asp_debug}), and, potentially, simulating
kernels or improving compilation speed. Separating these concerns
makes the original code easier to maintain and enables the automatic
application of these transformations to a wide range of designs,
improving developer productivity. Simulation aspects could be applied
to dataflow designs to generate equivalent state-based C code thus
enabling pure software simulation. Compilation aspects, on the other
hand, could be applied during the development process to create
versions of the dataflow design that compile faster by reducing the
operating frequency, removing debug blocks or applying design-level
optimisations that can resolve timing constraints. Naturally, reducing
the compilation time would increase developer productivity.


\begin{table}[tp]
\caption{Types of Aspects used in \MAXC{}}
\renewcommand{\arraystretch}{1.5}
\label{tbl:aspects}
\centering
\begin{tabular}{l|l|l}
\hline
\bf{Aspect Type} & \bf{Aspect Name} & \bf{Description} \\
\hline
\hline
\multirow{3}{*}{system} & \blt hw/sw partitioning & capture mapping between  \\
                        & \blt monitorisation & application modules and \\
                        & \blt reconfiguration & GPP/FPGA accelerators\\
\hline
\multirow{2}{*}{implementation} &\blt operator optimisation &  capture low-level hardware \\
& \blt word-length spec & optimisations  \\
\hline
\multirow{3}{*}{exploration} & \multirow{2}{*}{\blt iterative} & generate multiple implemen- \\
 & \multirow{2}{*}{\blt metaheuristic} & tations based on design  \\
 & & space exploration strategies \\
\hline
\multirow{3}{*}{development} & \blt simulation & \multirow{2}{*}{improve developer}  \\
& \blt debugging & \multirow{2}{*}{productivity} \\
& \blt compilation &  \\
\hline
\end{tabular}
\end{table}

\subsection{Hardware/software partitioning}
\label{sect:asp_hsp}
\MAXC{} functions describing dataflow computations can be embedded
within the C application but cannot be invoked directly by software C
functions.  Instead, a \MAXC{} pragma must be used on top of software
function definitions or C calls to indicate an alternate hardware
implementation. For instance, the code shown in
Fig. \ref{fig:aspect-switch} indicates that the software
implementation of \texttt{f()} can be mapped to the dataflow
implementation described in \texttt{fast\_f()}. This way, our
design-flow can automatically switch from a pure software application
to a software/hardware design.

\lstset{style=MaxC}
\begin{figure}[!h]
\begin{lstlisting}
(*@\marktext{void fast\_f()}@*) {/* dataflow implementation */}

void      f() {/* software implementation */}

(*@ \marktext{\#pragma FAST hw:fast\_f} @*)
f();
\end{lstlisting}
\caption{Mapping of C function calls to dataflow kernels using \MAXC{} pragmas.}
\label{fig:aspect-switch}
\end{figure}


\noindent

Hence, a hardware/software partitioning strategy can be performed in
five steps:
\begin{enumerate}
  \item detecting hotspots in the program;
  \item detecting code patterns from hotspots that are suited for
    dataflow computation and acceleration;
  \item performing the outlining transformation so that each
    candidate for acceleration is enclosed in a function \texttt{f};
  \item deriving a dataflow version \texttt{fast\_f} from state-based
    \texttt{f};
  \item placing a FAST pragma on top of each function call to
    \texttt{f} and associate it to the corresponding \texttt{fast\_f}
    function.
\end{enumerate}

Each of these steps can be described as a separate LARA aspect and
combined to form a hardware/software partitioning strategy.

\subsection{Monitorisation Aspect}
\label{sect:asp_mon}
To find potential hotspots in the application, for instance to perform
hardware/software partitioning, we can use the aspect in
Fig.~\ref{fig:hotspot}.  With this aspect, the weaver can
automatically instrument any C application to self-monitor its
innermost loops at run-time, as they are natural candidates for
dataflow-based acceleration. In particular, this monitorization aspect
can compute the following information for every innermost loop: (a)
the average number of times it has been executed, (b) the average
number of iterations, (c) the loop average time, and (d) the loop
iteration average time. For this purpose, we use a monitoring API
composed of 4 functions to mark the beginning and end of the loop
(\texttt{monitor\_instanceI} and \texttt{monitor\_instanceE}
respectively), and to mark the beginning and end of an iteration
(\texttt{monitor\_iterI} and \texttt{monitor\_iterE}
respectively). These monitoring functions keep an account of the
frequency of execution and the time to complete the whole loop and a
single iteration.

\lstset{style=lara}
\begin{figure}[!h]
\begin{lstlisting}
aspectdef LoopMonitor
select function.loop{is_innermost} end
apply
    $loop.insert before
       %{monitor_instanceI("[[$loop.key]]");}%;
    $loop.insert after
       %{monitor_instanceE("[[$loop.key]]");}%;
end

select function.loop{is_innermost}.entry end
apply $begin.insert after
       %{monitor_iterI("[[$loop.key]]");}%;
end
select function.loop{is_innermost}.exit end
apply $begin.insert before
       %{monitor_iterE("[[$loop.key]]");}%;
end
end
\end{lstlisting}
\caption{Aspect that instruments the application to monitor loop
  activity. The information generated can be used to identify
  hotspots.}
\label{fig:hotspot}
\end{figure}

  The aspect code is as follows: line~2 selects all
loops in the application that are innermost (loops with no other loops
enclosed); lines~3--8 place an instance monitor call before and after
each selected loop; lines 10--13 select all entry points inside the
loop and insert a monitoring call to mark the beginning of each
iteration; lines~14--17 place an instance monitor call to mark the end
of each iteration. The following table shows an example of applying
the aspect from Fig.~\ref{fig:hotspot} on a C-style function containing a loop:
\vspace{2mm}

{\footnotesize
\fontfamily{pcr}\selectfont
\begin{tabular}{l|l}
\hline
\bf{original code}           & \bf{woven code}                                   \\
\hline
\hline
void f() \{                  & void f() \{                                       \\
%\hspace{3ex}$\ldots$        & \hspace{3ex}$\ldots$                              \\
                             & \hspace{3ex}\marktext{monitor\_instanceI("f:1");} \\
\hspace{3ex}while (i < N) \{ & \hspace{3ex}while (i < N) \{                      \\
                             & \hspace{6ex}\marktext{monitor\_iterI("f:1");}     \\
\hspace{6ex}i++;             & \hspace{6ex}i++;                                  \\
                             & \hspace{6ex}\marktext{monitor\_iterE("f:1");}     \\
\hspace{3ex}\}               & \hspace{3ex}\}                                    \\
                             & \hspace{3ex}\marktext{monitor\_instanceE("f:1");} \\
\}                           & \}                                                \\
%\hspace{3ex}$\ldots$        & \hspace{3ex}$\ldots$                              \\
\hline
\end{tabular}
}
\vspace{2ex}

\noindent Each monitoring call in the woven code receives as a
parameter the loop key, which uniquely identifies the loop within the
application. The loop key is generated by concatenating the function
name with the hierarchical position of the loop within the abstract
syntax tree. For instance, \emph{f:2:1} corresponds to the 1st loop
inside the 2nd outermost loop of function \texttt{f}. The hotspots can
be identified by an aspect (not shown) that takes the profiling
information generated by the monitorization API calls, and that uses
an heuristic to compute the most profitable computations to be
offloaded to hardware.

\begin{comment}
\subsection{Reconfiguration Aspect}
\label{sect:asp_reconfig}
To support run-time reconfiguration, we specify the configuration
associated with the function call in the \MAXC{} pragma. For instance:

\vspace{3mm}
\noindent\texttt{\footnotesize{\marktext{\#pragma FAST hw:fast\_f0 cfg:c0}\\
x = f(0); \\
\marktext{\#pragma FAST hw:fast\_f1 cfg:c1}\\
y = f(x); \\
\marktext{\#pragma FAST hw:fast\_g cfg:c1}\\
z = g(x); \\
}}

\noindent With the above code annotations, our design-flow can
generate multiple configurations, each containing a set of \MAXC{}
implementations that can be executed in parallel. If the configuration
name is not specified using the \MAXC{} pragma, then we assume a
default configuration. Having a single configuration can lead to
situations where at any point in time and due to data dependencies,
part of the functions are idle. With run-time reconfiguration, we can
exploit unused resources to support active functions. In particular,
during the execution of an application, we select various
configurations at different points in time to maximise the utilisation
of FPGA resources. Within this context, we use a hardware partition,
which contains a set of configurations that are used to support
reconfiguration during the life cycle of an application. In the above
example, configuration \texttt{c0} contains a single implementation of
\texttt{f} (\texttt{fast\_f0}), and thus can potentially use more
resources and be faster than the \texttt{fast\_f1} version which must
share the same configuration (\texttt{c1}) with \texttt{fast\_g}.

The work in~\cite{Xinyu:Qiwei:Luk:Qiang:Pell:2012} proposes an
approach for extracting valid and efficient hardware partitions. To realize
run-time reconfiguration without modifying the original code we use the
aspect shown in Fig.~\ref{fig:aspect-reconf}.  The input to the
aspect is a hardware partition (lines 2--4). The partition is implemented as a
hash table that maps a function call (key) to a hardware
implementation, represented as a tuple containing the hardware
implementation name (hw) and the associated configuration (cfg).

\lstset{style=lara}
\begin{figure}[!h]
\begin{lstlisting}
aspectdef AspReconfig
input
   partition
end
select function.call end
apply
   if ($call.key in partition) {
      var cfg = partition[$call.key].cfg;
      var hw = partition[$call.key].hw;
      $call.insert before %{
         #pragma FAST hw:[[hw]] cfg:[[cfg]]
      }%;
   }
end
end
\end{lstlisting}
\caption{Reconfiguration aspect.}
\label{fig:aspect-reconf}
\end{figure}

Table \ref{fig:aspect-hash} shows an example of a hash table
representing a hardware partition. The key (e.g. main:f:1) identifies
a function call in the application, and is formed by concatenating the
name of the caller function (main), the name of the invoked function
(e.g. f) and a unique number (1).  Line~5 in the aspect shown in
Fig.~\ref{fig:aspect-reconf} selects all function calls, and for each
call found in the input partition (line~7), we set the appropriate
pragma on top of the call statement (lines 10--12). We can now realize
and experiment different reconfiguration designs by invoking this
aspect with different hardware partitions.


%{\footnotesize
%\fontfamily{pcr}\selectfont
\begin{table}[!h]
\caption{An example of a hardware partition, represented as a hash
  table, used with the reconfiguration aspect
  (Fig.~\ref{fig:aspect-reconf})}
\label{fig:aspect-hash}
\centering
\begin{tabular}{c|c|c}
\hline
\multicolumn{3}{c}{\bf{partition}} \\
\hline
\bf{\$call.key} & \bf{hw} & \bf{cfg}  \\
\hline
main:f:1 & fast\_f0 & c0 \\
main:f:2 & fast\_f1 & c1 \\
main:g:3 & fast\_g & c1 \\
\hline
\end{tabular}
\end{table}
%}

\end{comment}

\subsection{Operator Optimisation Aspect}
\label{sect:asp_ops}
To provide architectural details to \MAXC{} designs, such as mapping
operators to DSP blocks, we can use the \MAXC{} pragma shown in
Fig. \ref{fig:aspect-balance} at the top of a statement (including code
blocks). The balancing parameter corresponds to the degree of
utilisation of DSP blocks in a statement.

\lstset{style=MaxC}
\begin{figure}[!h]
\begin{lstlisting}
(*@ \marktext{\#pragma FAST balanceDSP:balanced} @*)
{
  x = x * y;
  x++;
}
\end{lstlisting}
\caption{The \MAXC{} balancing pragma provides fine grained control
  over the mapping of computation to either DSPs or LUT/FF pairs.}
\label{fig:aspect-balance}
\end{figure}

The aspect shown in Fig.~\ref{fig:aspect-DSP} is the strategy for
balancing DSP blocks in every statement of an application. Instead of
adding the above pragma manually, we provide a set of rules
(lines~3--4) that define where to place the \texttt{balanceDSP}
pragma. In this example, we established the rule that full DSP block
utilisation is applied to any statement that has 5 or more
multipliers and adders, balanced if 3 or more multipliers, and no DSP
utilisation otherwise.

\lstset{style=lara}
\begin{figure}[!h]
  \centering
  \begin{lstlisting}
aspectdef DspBalancing
var op_granularity =
 [{DspBalance:'full',MultiplyOp: 5,AddOp: 5 },
  {DspBalance:'balanced',MultiplyOp:3}];

select function.statement end
apply
   for (var i in op_granularity) {
      var gprofile = op_granularity[i];
      var match = true;
      for (var k in gprofile) {
         if (k != 'DspBalance') {
            match &= ($statement.num_construct(k)
                      >= gprofile[k]);}}
      if (match) {
         var pragma = '#pragma FAST balanceDSP:'
                      + gprofile.DspFactor;
         $statement.insert before "[[pragma]]";
         break;}}
   end
end
  \end{lstlisting}
  \caption{Aspect for exploring mapping of computation to DSP blocks.}
  \label{fig:aspect-DSP}
\end{figure}

\subsection{Iterative Aspect}
\label{sect:asp_it}
Using LARA we can implement and combine these aspects to enable
systematic design space exploration of all the optimisation options
exposed by the \MAXC{} backend resulting in the generation of a large
number of designs. The feedback-directed compilation process of LARA
can be used to capture and extract feedback from the backend reports
pertaining to resource usage or timing information and automatically
adjust the compilation process.

An example of a LARA aspect for design space exploration is
shown in Fig.~\ref{fig:aspect-exploration}. It highlights the feedback capabilities of the design
flow: the aspect will generate and build the \MAXC{} designs until the
resource usage passes a specified LUT threshold, and at each step
increasing a particular design attribute, such as exponent, mantissa or the parallelism of the design (by replicating the computational pipeline).

\lstset{style=lara}
\begin{figure}[!h]
\begin{lstlisting}
  aspectdef DesignExploration
  input
     attribute,
     start, step,
     lut_threshold,
     config
  end
  config[attribute] = start;
  var i = 0;
  do {
    var designName = genName(config);
    call genFAST(designName, config);
    buildFAST(designName);
    config[attribute] += step; i++;
  } while (@hw[designName].lut < lut_threshold
           && i < LIMIT);
  end
\end{lstlisting}
\caption{Exploration aspect that generates multiple \MAXC{} designs by varying a design attribute (e.g. number of kernels or mantissa) until a LUT threshold is reached.}
\label{fig:aspect-exploration}
\end{figure}


\subsection{Debugging Aspect}
\label{sect:asp_debug}
Because the current execution model does not provide run-time debugging
of hardware designs, the easiest solution to debug designs is to log
the values of various streams during execution. The insertion of debug
statements can be encapsulated in aspects. It is particularly
important to separate debug aspects from the original application code
since debug blocks can influence the compilation time and timing
constraints as well as the behaviour of the design. As an example,
the aspect in Fig. ~\ref{fig:aspect-debug} instruments the code to log every change in the value of a variable.

\lstset{style=lara}
\begin{figure}[!h]
  \centering
\begin{lstlisting}
aspectdef WatchVar
select function.vref end
apply
   $vref.parent_stmt.insert before
    %{ log("[[$vref.name]]", [[$vref.name]]); }%
   $vref.parent_stmt.insert after
    %{ log("[[$vref.name]]", [[$vref.name]]); }%
end
condition $vref.is_out end
end
\end{lstlisting}
  \caption{Aspect for automatically instrumenting the code to watch any change in the value of a program variable.}
  \label{fig:aspect-debug}
\end{figure}
