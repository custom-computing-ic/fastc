\begin{abstract}

  The dataflow model of computation is a convenient paradigm for
  expressing high-throughput parallel computations. It enables
  synthesis of deeply pipelined hardware architectures directly from a
  program's dataflow graph. Dataflow designs can be conveniently
  implemented on Field Programmable Gate Arrays which combine the
  advantage of fully customizable logic with low time to
  market. However, the manual development process used until now is
  challenging, error-prone and counter-productive. We propose a design
  flow based on aspect-oriented programming to decouple design
  development from design optimisation, thus improving
  maintainability, portability and developer productivity while
  enabling automated exploration of design trade-offs that can lead to
  increased performance. The proposed approach uses MaxC, a novel
  language for specifying dataflow designs. In this paper we introduce
  MaxC and show how it can be used to specify high performance
  dataflow designs. Optimisation strategies for the generated designs
  are then specified with LARA, a domain-specific aspect-oriented
  programming language, and MaxC. We evaluate our approach by
  developing an implementation for a high performance application
  based on Reverse-Time Migration (RTM). Preliminary results show that
  our design achieves comparable performance to state of the art FPGA
  implementations, while improving developer productivity.

\end{abstract}
