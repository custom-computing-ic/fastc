\section{Conclusions}

\begin{frame}{Current and Future Work}

  \begin{itemize}
  \setlength{\itemsep}{10pt}

  \item Finish second paper for FPT'13: aspect descriptions and tools
    for run-time reconfiguration

  \item Cover other classes of parallel computation

  \item Cover heterogeneous systems

  \item Support other frontend languages

  \item Cover other applications domains

  \item Improved run-time support (CPU -- DFE interface)

  \item Automated translation to dataflow designs

  \item Validate improved productivity and
    portability claims

  \end{itemize}
\end{frame}


\begin{frame}{Summary}

  \begin{beamerboxesrounded}{Aspect-driven Approach}
    \begin{itemize}
    \item improve productivity by encapsulating optimisations
    \item improve efficiency by effective design space exploration
    \end{itemize}
  \end{beamerboxesrounded}

  \begin{beamerboxesrounded}{FAST Dataflow Language}
    \begin{itemize}
    \item simple, intuitive specification of dataflow designs
    \item integrate AOP techniques with dataflow compilation tools
    \end{itemize}
  \end{beamerboxesrounded}

  \begin{beamerboxesrounded}{Aspect Descriptions}
    \begin{itemize}
      \item improve performance, productivity; automate exploration
      \item support for run-time reconfiguration
    \end{itemize}
  \end{beamerboxesrounded}

  \begin{beamerboxesrounded}{Demonstrate Automated Design Flow}
    \begin{itemize}
    \item produce fastest and most energy efficient RTM design
    \item up to 80\% code reduction, 4 -- 16 times less API calls
    \item best paper candidate at ASAP 2013 (4 of 125 submissions)
    \end{itemize}
  \end{beamerboxesrounded}

\end{frame}
