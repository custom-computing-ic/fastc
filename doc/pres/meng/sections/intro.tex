\section{Introduction}

\begin{frame}{High Performance Computing}
  \vspace{0.5cm}
  \begin{beamerboxesrounded}{Applications}
    \begin{columns}[c]
      \column{.3\textwidth}
      \vspace{-1.2cm}
      \begin{itemize}
      \item Oil and Gas
      \item Finance
      \item Online Ads
      \end{itemize}
      \column{.6\textwidth}
      \vspace{-1.2cm}
      \begin{columns}
        \column{.5\textwidth}
        \begin{figure}[ht]
          \centering
          \includegraphics[scale=0.18]{figs/oil-gas-industry.jpg}\\
        \end{figure}
        \column{.5\textwidth}
        \begin{figure}[ht]
          \centering
          \includegraphics[scale=0.29, clip=true, trim=0 0 0 0]{figs/finance.jpg}\\
        \end{figure}
      \end{columns}
      \vspace{-0.45cm}
      \begin{figure}[ht]
        \centering
        \includegraphics[scale=0.25]{figs/ads.jpg}\\
      \end{figure}
    \end{columns}
  \end{beamerboxesrounded}
  \vspace{0.2cm}
  \begin{beamerboxesrounded}{Requirements}
    \begin{columns}
      \column{.3\textwidth}
      \begin{itemize}
        \centering
      \item Speed
      \end{itemize}
      \column{.35\textwidth}
      \centering
      \begin{itemize}
        \centering
      \item Energy Efficiency
      \end{itemize}
      \column{.3\textwidth}
      \begin{itemize}
      \centering
      \item Productivity
      \end{itemize}
    \end{columns}
  \end{beamerboxesrounded}
\end{frame}




\begin{frame}{This Project}
  \begin{beamerboxesrounded}{Objectives}
    \begin{itemize}
    \item more efficient HPC -- using dataflow computing
    \item more productive HPC -- using aspect-oriented design
    \end{itemize}
  \end{beamerboxesrounded}
  \vspace{0.3cm}
  \begin{beamerboxesrounded}{Impact}
    \begin{itemize}
    \item first aspect-oriented compilation flow for dataflow HPC
    \item produce fastest, most energy efficient RTM dataflow design
    \item up to 80\% code reduction, 4 -- 16 times less API calls
    \item techniques deployed in FP7 HARNESS project
    \item best paper candidate at ASAP 2013 (4 of 125 submissions)
    \end{itemize}
  \end{beamerboxesrounded}
\end{frame}


\begin{frame}
  \frametitle{Dataflow High-Performance Computing}
  \begin{figure}[!ht]
    \centering
    \def\svgwidth{0.9\linewidth}
    \input{figs/dataflow-both.pdf_tex}
  \end{figure}
\end{frame}

\begin{comment}
  \begin{frame}
    \frametitle{CPU vs Dataflow}
    \begin{columns}
      \begin{column}{.5\linewidth}
        \vspace{-1cm}
        \begin{figure}[!ht]
          \centering
          \def\svgwidth{\linewidth}
          \input{figs/compute-cpu.pdf_tex}
        \end{figure}
        \vspace{-0.5cm}
        \begin{itemize}
        \item Fast to develop
        \item Inefficient
        \end{itemize}
      \end{column}
      \begin{column}{.5\linewidth}
        \vspace{-1cm}
        \begin{figure}[!ht]
          \centering
          \def\svgwidth{\linewidth}
          \input{figs/compute-dfe.pdf_tex}
        \end{figure}
        \vspace{-0.5cm}
        \begin{itemize}
        \item Slow to develop
        \item Efficient
        \end{itemize}
      \end{column}
    \end{columns}
  \end{frame}
\end{comment}
\begin{frame}[fragile]
  \frametitle{Kernel Optimisations Space}
  \begin{figure}[!ht]
    \centering
    \def\svgwidth{\linewidth}
    \input{figs/dfg-opt-all.pdf_tex}
  \end{figure}
\end{frame}

\begin{frame}[fragile]
  \frametitle{Dataflow High-Performance Computing}
  \begin{comment}
    Developers must manually transform the original design to:
    \begin{enumerate}
    \item ensure I/O interface compatibility with host
    \item optimise resource usage by minimising operand width
    \item ensure result correctness by casting
    \item apply FPGA and dataflow specific optimisations
    \item replicate computational pipelines
    \end{enumerate}
    \vspace{0.5cm}
  \end{comment}
  \begin{beamerboxesrounded}{The Problem}
    Mixing optimisations with functional code makes it:
    \begin{itemize}
    \setlength{\itemsep}{10pt}
    \item harder to infer optimal designs (more constraints)
    \item harder to automate design space exploration
    \item impossible to re-target optimisations automatically
    \end{itemize}
  \end{beamerboxesrounded}
\end{frame}

\begin{comment}
  \begin{frame}[fragile]
    \frametitle{Decoupling Optimisations}
    \framesubtitle{Aspect-Oriented Design}
    \begin{beamerboxesrounded}{Aspect-Oriented Programming (AOP)}
      \begin{itemize}
      \item encapsulate cross-cutting concerns in aspect descriptions
      \item apply them at run-time to alter behaviour
      \item apply them at specified points (pointcut model)
      \end{itemize}
    \end{beamerboxesrounded}
    \vspace{0.75cm}
    \begin{beamerboxesrounded}{Aspect-Oriented Programming for Dataflow Engines}
      \begin{itemize}
      \item capture aspects of quality and their realisation
        \begin{itemize}
        \item speed, area, power, energy
        \item numerical accuracy, parallel computation
        \item optimisation, monitoring, debugging
        \end{itemize}
      \item use aspects to generate multiple variants of a design
      \end{itemize}
    \end{beamerboxesrounded}
  \end{frame}
\end{comment}

\begin{frame}[fragile]
  \frametitle{Decoupling Optimisations: Aspect-Oriented Design}
  \begin{figure}[!ht]
    \centering
    \def\svgwidth{\textwidth}
    \input{figs/meng-aspect-weaving.pdf_tex}
  \end{figure}
\end{frame}
