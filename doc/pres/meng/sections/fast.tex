\section{FAST}

\begin{frame}[fragile]
  \frametitle{2. FAST -- Facile Aspect-driven Source Transformation}

  \begin{lstlisting}[style=MaxC,   basicstyle=\scriptsize]
    // kernel defines input, output streams and scalars
    void kernel_MovingAvg(float* in, float* avg, int bound)
    {
      // special macros implemented as counters
      bool hasPrev = CURRENT_CYCLE >=bound;
      bool hasNext = CURRENT_CYCLE < MAX_CYCLE - bound;
      bool hasBoth = hasPrevious & hasNext;

      // ternary operator, stream offsets, arithmetic
      float prev = hasPrev ? in[-1] : 0.0;
      float next = hasNext ? in[ 1] : 0.0;
      float sum  = prev + in[0] + next;

      // output assignment
      avg[0] = hasBoth? sum / 3 : sum / 2;
    }

    int main() {
      #pragma fast hw:MovingAvg  // maps to dataflow kernel
      MovingAvg_CPU(...);        // call regular C function
    }
  \end{lstlisting}
\end{frame}

\begin{frame}
  \frametitle{FAST Features}

  {\footnotesize
    \begin{table}[!h]
      \centering
      \renewcommand{\arraystretch}{1.3}
      \hspace*{-2pt}\makebox[\linewidth][c]{
      \begin{tabular}{p{3cm}|p{3cm}|p{1.1cm}|p{3cm}}
        \hline
        \bf{Feature}                        & \bf{Implementation}              & \bf{Method}   & \textbf{Compared to C}      \\
        \hline\hline
        Dataflow Kernels & Function definition & C99 & Cyclic execution \\
        \hline
        I/O                               & Kernel arguments              & Inferred        &  Only params, =  \\
        \hline
        Control                           & Ternary op. (?:), \texttt{if} & C99             &     \\
        \hline
        \multirow{2}{*}{Computation}        & log, exp, sqrt, sin etc.                     & math.h           & Pragmas for range      \\
        &  +, *, /, -     & C99         \\
        \hline
        \multirow{2}{*}{Streams}          & Declared as pointers          & \multirow{2}{*}{C99} &  \multirow{2}{3cm}{Cyclic update; negative offsets } \\
        & Array index access     &                      \\
        \hline
        Optimisation                      & \multirow{2}{*}{C pragmas}    & \multirow{2}{*}{C99} & \multirow{2}{*}{Run-time values}\\
        Hardware \  Mapping               &                               &                      & \\
        \hline
        \multirow{2}{*}{Parameterization} & Constants, variables,         & \multirow{2}{*}{C99}  & \multirow{2}{*}{Compile-time bounds} \\
        & \texttt{for}, \texttt{while}  &                     &  \\
        \hline
      \end{tabular}
      }
    \end{table}
  }
  \begin{itemize}
  \item simplify integration with existing applications and tools
  \item simplify translation from pure C/C++ applications
  \end{itemize}
\end{frame}

\begin{comment}
\begin{frame}{2. FAST}
  \frametitle{FAST vs C}
  \begin{itemize}
  \item FAST only uses C \emph{syntax}
    \begin{itemize}
      \setlength{\itemsep}{5pt}
    \item to facilitate integration
    \item to express dataflow designs in a simple, intuitive fashion
    \end{itemize}
    \vspace{0.5cm}
  \item  Differences include:
    \begin{itemize}
      \setlength{\itemsep}{5pt}
    \item execution model only supports kernels
    \item pointers are regarded as   streams
    \item negative offsets are allowed
    \item only compile time loop bounds are allowed
    \item direct interoperability with C code is not possible, but
      simulated via pragmas
    \end{itemize}
  \end{itemize}
\end{frame}
\end{comment}